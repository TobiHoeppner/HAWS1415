\section{Fortsetzung UNION-FIND-Laufzeitanalyse  \tiny (Vorlesung 28 am 02.02.)}
$cost(\mathcal{C}) \leq j* \underbrace{l(\mathcal{C})}_{m}+2nL_j(r)$\\
$j=?$ Beide Terme ausgleichen: $jm \approx 2nL_j(r)$\\
wähle $j$ sehr klein: $\frac{m}{n} \approx L_j(r)$\\
$\alpha(m,n) = \min \{ j \geq 0 | L_j(\lfloor \log_2 n\rfloor) \leq 2 + \frac{m}{n}\}$

\subsection{Laufzeit einer Folge von UNION-FIND-Operationen}
$\mathcal{O}(m+n+ \underbrace{ cost(\mathcal{C})}_{\leq \underbrace{j*m}_{m*\alpha(m,n)} + 2n \underbrace{L_j (\lfloor \log_2 n \rfloor)}_{\leq 2+\frac{m}{n}}})$\\
$=\mathcal{O}(m+n+m\alpha (m,n)+4n+2m) = \mathcal{O}(m+n \alpha(m,n))$\\

$\L_0(r) = \lceil \log_2 r \rceil$\\
$\L_{j+1}(r) = \begin{cases} 0, &r\leq 1 \\1+L_{j+1}(L_j(r)),&r \geq 2 \end{cases}$\\

Umkehrfunktion: $L_j(r) \leq k \Leftrightarrow r \leq B_j(k)$\\
Abriss über große Zahlen bspw. $B_2(4) = 2 \uparrow(2^{16})$\\
$B_{j+1}(k) = B_j(B_{j+1}(k-1)$\\
Satz: Eine Folge von m FIND-Operationen und $\leq n-1$ UNION-Operationen auf einer Grundmenge von $n$ Elementen dauert bei Vereinigung nach Rand- und Pfadkompression. $\mathcal{O}(n+m \alpha(m,n))$ Zeit. (scharfe Schranke)\\

\subsection{einparametrige inverse Ackermannfunktion}
$\alpha(n) = \min\{j | L_j(n) \leq 3 \}$\\

Wo tritt diese Funktion auf?\\
%Bild 1
Algorithmus zum finden von zusammenhängenden Teilstücken in einer Menge von Intervallen. Laufzeit $\mathcal{O}(n \alpha(n))$

\section{Randomisierte Algorithmen und Datenstrukturen}
\subsection{Randomisierte Suchbäume(Balden, engl. treap)}
Quicksort: Laufzeit $\mathcal{O}(n^2)$\\
(erwartete Laufzeit bei zufälliger Pivotauswahl = $\mathcal{O}(n \log n)$.)\\
Sortieren mit einem Suchbaum: Laufzeit $\mathcal{O}(n^2)$, aber einfügen in zufälliger Reihenfolge: $\mathcal{O}(n \log n)$, was beim löschen wieder verloren geht. \\
Daher folgende Idee: Wir wollen einen Suchbaum so verwalten, als ob er durch einfügen in zufälliger Reihenfolge entstanden wäre.\\
Definition: Eine Balde ist ein binärer Suchbaum, bei dem jeder Knoten zusätzlich zum Schlüssel $x$ eine zufällig gewählte Priorität $y$ hat. Die Struktur einer Balde ist so, als ob die Knoten in nach $y$ sortierter Reihenfolge in einem binären Suchbaum eingefügt worden wären.\\
Wurzel = Knoten mit kleinstem $y$-Wert.\\
%Bild3
Nach $x$ Suchbaumeigenschaft.\\
Nach $y$ Haldeneingeschaft: Jeder $y-$Wert ist kleiner als seine Kinder (daher der kleinste $y$-Wert im Teilbaum)\\
Beispiel: $ x = 1, 4,9,17,23,40$\\
$y = .30,.20,.77,.16,.51,.46$\\
%BILD IV
\subsubsection{Einfügen}
Einfügen von $x = 8, y = 0,17$\\
%BILD V
Zunächst mit $y = \infty$ einfügen, als Blatt. Anschließend graduell auf den neu erzeugten Zufallswert $y$ reduzieren, wenn $y < y_{\text{Eltern}}$ und $\rightarrow$Rotation im Baum.
% BILD VI
\subsubsection{Löschen}
Vergrößere (in Gedanken) den $y$-Wert des zu lösenden Knotens, bis er ein Blatt wird. Dann einfach löschen.\\
%BILD 7, 8
\subsubsection{Analyse}
Schlüssel $x_1 < x_2 < ... < x_n$ ... Balde, $y_1,...,y_n$ zufällig.\\
Ein neuer Schlüssel $x \in (x_k, x_{k+1})$ soll eingefügt werden. $k=0,..,n, x_0 = -\infty, x_{n+1} = +\infty$\\
Laufzeit =$\mathcal{O}$(Weglänge beim einfügen von $x$ als Blatt) $+ \mathcal{O}$(Anzahl Rotationen danach); 2.Term ist durch den 1.Term beschränkt.\\
%Bild 9
$A = \#$Knoten auf dem Wege, für die $x$ in linken Teilbaum gesucht wird.\\
$A = \#$Knoten auf dem Wege, für die $x$ in rechten Teilbaum gesucht wird.\\
Gesucht: $E(A+B)$\\
Was muss passieren, dass Knoten $x_j$ bei A gezählt wird?\\
\begin{enumerate}
\item $j \geq k+1$
\item kein Knoten $x_{k+1},x_{k+2},...,x_{j-1}$ hat einen kleineren $y$-Wert als $x_j$\\
$P(x_j$ wird bei A mitgezählt$)$: $\leq P [y_i$ist das Mini$\{y_{k+1}, y_{k+2},...,y_y  \}] = \frac{1}{j-k}$\\
$I_j = \begin{cases}1, &x_j \text{(wird bei A mitgezählt)} \\0, &\text{sonst.}\end{cases}$\\
$E(A) = E(I_{k+1}+I_{k+2}+...+I_n)$\\
$=E(I_{k+1}),+...+E(I_n)$\\
$=P[I_{k+1}+...+P[I_n=1]] = 1+ \frac{1}{2} + \frac{1}{3} + ... + \frac{1}{n-k}$\\
\end{enumerate}  