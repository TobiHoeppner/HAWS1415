\section{Weitere Reduktionen \tiny (Vorlesung 23 am 16.01.)}
\subsection{Graphenfärbung}
Eingabe: Graph $G = (V,E)$\\
Frage: Lässt sich $G$ mit $k$ Farben färben?\\
$\exists g: V \rightarrow\{1,...,k\}: \forall uv \in E : g(u) \neq g(v)$?\\
\subsection{3-Färbbarkeit}
wie oben, nur mit $k=3$ fest.\\
Satz: Die 3 Färbbarkeit ist NP-vollständig.\\
\begin{enumerate}
\item $\in$ NP
\item 3-SAT $\leq_p$ 3-Färbbarkeit
\end{enumerate}
\subsubsection{Vorrichtungen (Gadgets)}
Wir definieren uns ein (logisches) Gatter:\\
% E------o
%				\
%				 o - Ausgang
%				/
% E------o
Lemma: 
\begin{enumerate}
\item Wenn die beiden Eingänge die gleiche Farbe haben, dann hat der Ausgang ebenfalls diese Farbe.
% A------B
%				\
%				 A - Ausgang
%				/
% A------C

% A------C
%				\
%				 A - Ausgang
%				/
% A------B
\item Wenn die Eingänge verschiedene Farben haben, dann kann der Ausgang jede beliebige Farbe haben:
% A------B
%				\
%				 A - Ausgang
%				/
% B------C

% A------C
%				\
%				 B - Ausgang
%				/
% B------A

% A------B
%				\
%				 C - Ausgang
%				/
% B------A
\end{enumerate}

Als nächstes bauen wir uns ein ODER-Gatter mit 3 Eingängen:

% E------o
%				\
%				 o----o
%				/		|
% E------o			|
%						|
% E--------------o - Ausgang

\subsubsection{Reduktion}
Gegeben Boolesche Formel $\psi$ in 3-SAT Form.\\
% Komplexer Graph mit Basisdreieck und Oder-Gatter

Für jede Klausel, erstelle ein ODER-Gatter mit den entsprechenden Literalten als Eingabeknoten. Der Ausgang des Gatters wird mit dem F-Knoten des Basisdreiecks verbunden. (Beispiel: $x_1 \lor \lnot{x}_2 \lor x_3$)\\
\subsubsection{Beweis}
Behauptung: $\psi$ erfüllbar $\Leftrightarrow G$ ist 3-färbbar. 
\begin{enumerate}
\item[$\Rightarrow$] erfüllende Bedingung\\
$x_i, \lnot x_i$ entsprechend Färben. Weil die Belegung alle Klauseln erfüllt, sind die Eingaben jedes Klauselgatters nicht (F,F,F), Man kann dann als Klauselgatter immer so färben, dass der Ausgang mit W gefärbt ist, das ist eine gültige Färbung von G.\\
\item[$\Leftarrow$] Färbung sei gegeben.\\
o.B.d.A. Basisdreiecke gefärbt mit W,F,O wie in Abbildung.
\begin{enumerate}
\item[$\Rightarrow$] $x_i, \lnot x_i$ sind mit W,F gefärbt, entgegengesetzt.
\item[$\rightarrow$] Belegung ablesen. Wenn eine Klausel nicht erfüllt ist, sind alle 3 Eingänge des entsprechenden ODER-Gatters F,F,F,...,F ist die einzige mögliche Farbe für den Ausgang. Widerspruch, weil Ausgang mit F verbunden ist.
\end{enumerate}
\end{enumerate}
$G = f(\psi)$ in polynomieller Zeit berechenbar. \checkmark \\
3-Färbbarkeit ist NP-vollständig sogar, wenn man es auf planare Graphen einschränkt.\\
\subsection{Reduktion von gewöhnlichen 3-FARB auf 3-FARB in planaren Graphen}
Skizze der Reduktion: Beliebige Zeichnung von $G$; Elimination der Kreuzungen.\\
% Siehe Bild mit 8 Ecken und Knotennamen rstu (als Eingänge)
In jeder gültigen Färbung des Achtecks bekommen $r$ und $t$, sowie $s$ und $u$ die gleichen Farben.\\

\subsection{Teilmengensumme (TMS)}
(Spezialfall des Rucksackproblems)\\
Eingabe: $n$ Zahlen $a_1,a_2,...,a_n \in \mathbb{N}$, Zielsumme $b$\\
Frage: gibt es eine $I \subseteq \{1,...,n\}$ mit $\sum_{i \in I}^{} a_i = b$\\
Satz: TMS ist NP-vollständig\\
\subsubsection{Beweis}
\begin{enumerate}
\item TMS $\in$ NP \checkmark
\item 3-SAT $\leq_p$ TMS\\
Gegeben Boolesche Formel $\psi$ mit $n$ Variablen und $m$ Klauseln:\\
$2n$ Literalzahlen $a_i,a_j$ (entsprechend $x_i,x_j$)\\
$a_1 = \underbrace{1}_{x_1}\underbrace{0}_{x_2}0...\underbrace{0}_{x_n}\underbrace{0000}_{c_1}\underbrace{0001}_{c_2}....\underbrace{0001}_{c_m}$, falls $x_1$ in Klauseln $2$ und $m$ vorkommt.\\
$b_1 = \underbrace{1}_{x_1}\underbrace{0}_{x_2}0...\underbrace{0}_{x_n}\underbrace{0001}_{c_1}\underbrace{0000}_{c_2}....\underbrace{0}_{c_m}$, falls $\lnot x_1$ in Klausel 1 vorkommt.\\
$b = \underbrace{1}_{x_1}\underbrace{0}_{x_2}0...\underbrace{0}_{x_n}\underbrace{0003}_{c_1}\underbrace{0003}_{c_2}....\underbrace{0003}_{c_m}$
\end{enumerate}
Für jede Klausel $j$ 2 Füllerzahlen:\\
$d_j=e_j=00000000\underbrace{0001}_{c_j}000$\\
Eingabe für TMS: $a_1,...,a_n,b_1,...,b_n,d_1,...,d_m,e_1,...,e_m$\\
$b$\\
Behauptung: $\psi$ erfüllbar $\Leftrightarrow$ TMS lösbar
\begin{enumerate}
\item[$\Rightarrow$] Belegung gegeben\\
wähle $d_i$ oder $b_i$ aus, je nachdem, ob $x_i = W$ oder $x_i = F$\\
In jeder Klauselspalte $j$ ist die Summe $1,2,3 = $(Anzahl der wartenden Literale)\\
Nimm $d_j + e_j$ bzw. $d_j$ bzw. gar nichts zur Lösung dann.\\
Summe = $111111...133333...3$
\item[$\Leftarrow$] Lösung in TMS gegeben. Bei Bildung der Summe gab es keinen Übertrag.\\
Gesamtsumme in den linken Spalten $=2$, in den rechten Spalten $=5 \Rightarrow$ Summenbedingung muss für jede Dezimalstelle gelten. $a_i$ oder $b_i$ ausgewählt, aber nicht beide. $\rightarrow$ Belegung.\\
Summe aller $a_i / b_i$ Variablen ist mindestens $1$ in jeder Klauselspalte. $\rightarrow$ jede Klausel ist erfüllt.\\ 
\end{enumerate}
\subsection{Rucksackproblem}
Das Ruckproblem ist NP-vollständig. Laufzeit $\mathcal{O}(nb)$ (dynm. Prog.)\\

Einen Algorithmus, der polynomiell in der Größe der Zahlen und nicht unbedingt in der Kodierungslänge der Zahlen (binär)ist, nennt man pseudopolynomiell.\\