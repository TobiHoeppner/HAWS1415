\section{Ausblick \tiny (Vorlesung 31 am 13.02.)}
\subsection{Gebiete}
\begin{enumerate}
\item Algorithmen auf Graphen
\item Kombinatorische Optimierung
\item Zeichenketten
\item Arithmetik
\item lineare Optimierung
\item Netzwerke, Flüsse in Netzen
\item Geometrie
\end{enumerate}
\subsection{Techniken}
\begin{enumerate}
\item randomisierte Algorithmen
\item Approximationsalgorithmen
\item andere Rechner-/Datenmodelle
\begin{enumerate}
\item parallele Algorithmen
\item externer Speicher
\item Datenströme
\end{enumerate}
\end{enumerate}

\subsection{Approximationsalgorithmen}
\subsubsection{Rundreise Problem}
\emph{2x Spannbaum-Heuristik.}
Annahme: Kostenmatrix erfüllt die Dreiecksungleichung $c_{ij}+c_{jk} \geq c_{ik}$\\
symmetrisch: $c_{ij}:= c_{ji} $\\
\begin{enumerate}
\item berechne kürzesten Spannbaum $T$\\
\item durchlaufe in einem geschlossenen Zyklus jede Kante von $T$ einmal in jeder Richtung.\\
\item[$\rightarrow$] Kreis, der jeden Knoten mindestens $1x$ besucht. Kosten = $2 c(T)$\\
\item Streiche mehrfache Besuche und kürze den Weg ab.\\
\item[$\rightarrow$] liefert eine Rundreise $R$ mit $c(R) \leq 2 c(R^{OPT})$ ($R^{OPT}$ - optimale Rundreise)\\
\end{enumerate}
Dieser Algorithmus ist eine sogenannte 2-Approximation. Verbessern kann man das mit einer $\frac{3}{2}$-Approximation\\
\begin{proof}
\begin{align*}
c(T) \leq & c(R^{OPT})\\
sogar & \leq c(R^{OPT}) - \frac{n-1}{n}\\
c(R) \leq 2c(T) \leq 2c(R^{OPT})\\
\end{align*}
\end{proof}
\subsubsection{Rucksackproblem}
\subsubsection*{dynamische Programmierung}
$\mathcal{O}(n*b)$ Zeit\\
Teilprobleme: maximaler Wert, der mit Gewicht $g$ aus den Gegenständen $1,...,i$ gebildet werden kann.
\subsubsection*{alternative}
$W = \sum w_i$ in $\mathcal{O}(n*W)$ Zeit.\\
Teilprobleme: kleinstes Gewicht, mit dem man aus den Gegenständen $1,...,i$ einen Wert $w$ bilden kann.\\
$= f_{i,w}$, wobei $i = 1,...,n$ und $w = 0,1,...,W$\\
Rekursion: $f_{i,w} = \min\{ f_{i-1,w},f_{i-1,w-w_i} +g_i\}$...mit Randbedingungen\\
Lösung: $\max\{ w | f_{n,w} \leq b\}$\\
Werte auf vielfache von $\alpha$ runden: $w_i \rightart \lfloor \frac{w_i}{\alpha}\rfloor - \alpha$ Rundungsfehler $< \alpha$\\
Laufzeit: $W' = \sum w'_i = \sum \lfloor \frac{w_i}{\alpha} = \sum \frac{w_i}{\alpha} = \frac{W}{\alpha}$, $w'_i \in \mathbb{Z}$.\\
Rundungsfehler $\rightarrow$ Fehler in der Lösung. $\leq n*\alpha$ (absoluter Fehler)\\
\subsubsection{Beispiel: Absolute Fehlerschwelle $\Delta$ vorgeben}
$\alpha$ ausreichen aus $n*\alpha \leq \Delta$.\\
relativer Fehler?\\
$\Delta = n*\alpha$ in Berechnung einsetzen zum Optimalwert $W^*$\\
$W_{\max} \leq W^* \leq n*w_{\max}$\\
$w_{\max}$ das größte Gewicht eines Gegenstandes $i$ mit $g_i \leq b$\\
Wir wollen eine $1-\epsilon$Approximation $S$: $W(S) \geq (1-\epsilon)W*^$(umgekehrt wie beim Rundreiseproblem, weil Maximum)\\
$\Delta \leq \epsilon*w_{\max}$ ist ausreichend.\\
$\Rightarrow \Delta \leq \epsilon W^*$... Das gerundete Gewicht der ursprünglichen Optimallösung ist:
\begin{align*}
w'(S^{OPT}) \geq & W^* - \Delta \Rightarrow \text{Approximationsalgorithmus liefert} S\\
\text{mit } \underbrace{ w'(S)}_{\text{gewichtetes Gewicht}} \geq W^* -\Delta \text{(optimiert das gerundete Gewicht!)}\\ 
w(S) \geq & w'(S) \geq W^* - \epsilon W^* = W^*(1-\epsilon)\\
\end{align*}
$\Delta := \epsilon w_{\max}$\\
$\alpha := \frac{\Delta}{n} = \frac{\epsilon * w_{\max}}{n}$\\
Laufzeit: $\mathcal{O}(\frac{n*W}{\alpha}) = \mathcal{O}(n*W * \frac{n}{\epsilon*w_{\max	}}) = \mathcal{O}(n*n*w_{\max}*\frac{n}{\epsilon*w_{\max}}) = \mathcal{O}(n^3 * \frac{1}{\epsilon})$\\
\begin{enumerate}
\item[-] polynomiell in $n$ und $\frac{1}{\epsilon}$ ($\epsilon$ = relativer Fehler)\\
Ein polynomielles Approixmationsschema (für jedes $\epsilon > 0$) PTAS
\item[-] sogar ein voll- polynomielles Approixmationsschema (auch polynomiell in $\frac{1}{\epsilon}$ ist) - FPTAS
\end{enumerate}