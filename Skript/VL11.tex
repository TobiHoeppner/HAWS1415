\section{Dynamische Programmierung (Fortsetzung)\tiny (Vorlesung 11 am 21.11.)}
\subsection*{Nachtrag: Isotone Regression}
\begin{align*}
f_k(z) &= \min \{ \sum_{i=1}^k  w_i |x_i - a_i| : x_1 \leq x_2 \leq ... \leq x_{k-1} \leq x_k = z \}\\
f_k(z) &=  \underbrace{\min \{\ f_{k-1}(\underbrace{ x}_{x_{k-1}}) | \underbrace{ x}_{x_{k-1}} \leq \underbrace{ z}_{x_{k}} \}}_{g_{k-1}(z)} + w_k |a_k-z|\\
\end{align*}
$g_{k-1}(z)$: 2 Fälle:
\begin{enumerate}
\item[$z\leq p_{k-1} \Rightarrow$] $g_{k-1}(z)=f_{k-1}(z)$; der optimale Wert von $x=x_{k-1}$ bei gegeben Wert von $z=x_k$ ist $z$ selbst.
\item[$z\geq p_{k-1} \Rightarrow$] der optimale Wert von x ist $p_{k-1}$ 
$g_{k-1}(z) = f_{k-1}(p_{k-1})$
\end{enumerate}
Wenn $x_k^*$ der optimale Wert von $x_k$ ist, dann ist der optimale Wert von $x_{k-1}$:\\
\begin{align*}
x_{k-1}^* = \min	\{ x_k^*, p_{k-1} \}
\end{align*}

%Bild
Darstellung einer stückweise linearen stetigen Funktion durch Differenzen von Steigungen.\\
%Bild mit y = s x + t und y = s' x +t 
\underline{Knick} an der \underline{Stelle} $x_0$ mit \underline{Knickwert} $s' - s$\\
Bei Addition einer linearen Funktion bleibt der \underline{Knickwert} unverändert! (und Knickstelle)\\

Die Funktion wird durch eine Folge von Knickwerten dargestellt. Jeder Knick ist ein Paar (\underline{Knickstelle}, \underline{Knickwert})\\
% Bild mit Folge von Knickwerten als Tupel angegeben.
Darstellung ...
\subsection{Algorithmus}
Übergang von $f_{k-1}$ zu $g_{k-1}$:\\
$(x,\Delta)$ sei der rechteste Knick:
\begin{lstlisting}
while s - \Delta \geq 0:
	s:=s-\Delta
	lösche den rechtesten Knick
p_{k-1}:=x (min von f_{k-1} gefunden)
neuen Knickwert des rechtesten Knicks = \Delta-s
s:=0
\end{lstlisting}
Übergang von $g_{k-1}$ zu $f_k$:
\begin{lstlisting}
Füge einen zusätzlichen Knick (a_k, 2_w_k) ein.
\end{lstlisting}

Die Knicke können als Prioritätswarteschlange $Q$ gespeichert werden. nach Schlüsselwert $x$ geordnet.

\begin{lstlisting}
Q = empty, s=0
for k = 1,...,n:
	Q.insert((a_k,2w_k))
	s = s + w_k
	(x,\Delta) := Q.findmax()
	while s - \Delta \geq 0:
		s = s - \Delta
		Q.deletemax()
		(x,\Delta) := Q.findmax()
	p_k = x
	ersetze \Delta in Q.findmax() durch \Delta-s
	s = 0
(Berechnung der Optimallösung)
x_n = p_n
for k = n-1, n-2, ... , 1
	x_k = min(x_{k-1}, p_k)
\end{lstlisting}
\section{Der optimale Suchbaum}
Schlüssel:
\begin{lstlisting}
	AARON, p_1
	KLAUS, p_3
	DIETER, p_2
\end{lstlisting}
Mögliche Suchbäume:\\
% Bild 
% A(P1)
%(Q0)			K(P3)
%		D(P2)

% oder Bild
% 				D
%	A					K

Schlüssel $x_1 < x_2 < ... x_n$\\
Anfragehäufigkeiten $p_1, p_2, ... ,p_n$ für die Schlüssel\\
und $q_0,q_1, ... , q_n$ für die Intervalle zwischen den Schlüsseln\\
% Bild
%																												  q0	   q1  v   q2		q3
% "Zeitstrahl" die Q-Werte stehen über dem Strahl mit Pfeil nach unten. -v--A--v---D--v---K--v-

Mittlere gewichtete Weglänge:
\begin{align*}
&= \sum_{i=1}^n p_i \underbrace{(\text{Tiefe des Knotens mit Schlüssel}  i)}_{\# \text{ Vergleiche } -1} + \sum_{i=0}^n q_i \underbrace{(\text{ Tiefe des Blattes, dass dem entsprechenden Intervall entspricht})}_{\#\text{Vergleiche}}
\end{align*}
soll minimiert werden.

Ansatz: $q_i$ entspricht dem Intervall ($x_i, x_{i+1}$) $x_0 = -\infty, x_{n+1} = +\infty$\\
Ein Teilbaum in einem Suchbaum entspricht einem Interval ($x_i, x_j$ mit $0\leq i \leq j \leq n+1$)\\
(Der Baum wird genau dann betreten, wenn der gesuchte Schlüssel in diesem Intervall liegt.)\\

\subsection{Teilprobleme}
 $f(i,j)$ $0\leq i \leq j \leq n+1$ optimaler Suchbaum für Schlüssel $x_{i+1}, ... , x_{j-1}$ und Häufigkeiten $p_{i+1},...,p{j-1}$\\
 
 \subsection{Rekursion}
 \begin{align*}
 f(i,j) &= \min \{ f(i,k) + f(k,j)\} + q_i + q_{i+1} + ... + q_j + p_{i+1} + p{i+2} + ... + p_{j-1} - p_{k} &| i+1\leq k \leq j-1, 0 \leq i, j \leq n+1, j \geq i+1\\
 \end{align*}
 
 \subsection{Anfangswerte}
 \begin{align*}
 f(i, i+1) = 0, i = 0,...,n
 \end{align*}
 % Bild: Baum mit 3 Knoten
 $f(2,4) = f(2,3) + f(3,4) + q_2 + q_3 + p_3 -p_3$
 
\subsection{Gesamtlösung}
$f(0,n+1)$ 
 
 \subsection{Laufzeit}
 $\mathcal{O}(n^2)$ Teilprobleme. \\
 $\mathcal{O}(n)$ Ausdrücke, über die minimiert wird.\\
 Die Summe $q_1 + ... + p_{i-1}$ ist fest, für jedes Teilproblem nur einmal ausrechnen $\rightarrow \mathcal{O}(n)$ Zeit.\\
 $\Rightarrow \mathcal{O}(n^3)$ Zeit, $\mathcal{O}(n^2)$.