\section{Dynamische Programmierung (Einen hab ich noch!)\tiny (Vorlesung 12 am 24.11.)}
\subsection{Dynamische Programmierung - eine Zusammenfassung}
Bisher gelöste Probleme mit dynm. Prog.:
\begin{enumerate}
\item Rucksackproblem
\item optimale Triangulierung $\mathcal{O}(n^3)$
\item Suchbaum $\mathcal{O}(n^3)$
\item CYK (zweites Semester GTI) - Coche, Younger, Kasamy\\
Eine Grammatik ist in Chomsky Normalform (CNF), wenn jede jede Formel einer Grammatik in ein Terminalsymbol "mündet".\\
Ziel ist es für ein gegebenes Wort zu prüfen, ob es von der gegeben CNF Grammatik erzeugt werden kann.\\
$f(i,j) = \{ v | v \rightarrow^s{x} a_i,...,a_j\} | a_i, a_j \in \Sigma^*$
\end{enumerate}
\subsection{Editierabstand}
Wir nehmen an, dass wir folgendes Wort auf der Tastatur getippt haben:\\
$\quad \quad $ A g o r h y t m u s\\
Aber eigentlich wollten wir Pfannku... ähm \textbf{Algorithmus} schreiben.\\ 
Wie kommen wir jetzt von Agorhytmus zu Algorithmus?\\
\subsubsection{Problem}
\textbf{Gegeben}: zwei Wörter $A = a_1,...,a_m$ und $B = b_1,...,b_n$ aus $\Sigma^*$\\
Wie können wir $A$ in $B$ durch folgende Operationen verwandeln?
\begin{enumerate}
\item löschen eines Buchstabens (Kosten $k_L$)
\item einfügen eines Buchstabens ($k_E$)
\item Buchstabe $u$ durch $x$ ersetzen ($\delta_{ux}$)
\end{enumerate}
\textbf{Gesucht} ist die billigste Folge von Operationen wie $A$ in $B$ umgewandelt wird.\\
Die \textbf{Kosten} $k_L, k_E, \delta_{ux}$ sind der \textbf{Editierabstand}.\\
Vor allem in der Bioinformatik ist das ein sehr wichtiges Problem auf sehr großen Datenmengen.\\
\subsubsection{Teilprobleme}
$f(i,j) =$ Editierabstand zwischen $a_1, ... , a_i$ und $b_1, ... , b_j$
\subsubsection{Rekursion}
\begin{align*}
f(i,j) &= \min\{ f(i-1,j) + k_l, f(i,j-1)+k_E, f(i-1,j-1) + \delta_{a_i,b_j}\} &| 1\leq i \leq m, 1 \leq j \leq n
\end{align*}
\subsubsection{Startwerte}
Konvention: $\delta_{xx} = 0 \forall x \in \Sigma$
\begin{align*}
f(i,0) &= f(i-1,0) + k_L = i*k_L\\
f(0,0) &= 0\\
f(0,j) &= j k_E\\
\end{align*}

\subsubsection{Graph}
Graph mit $(m+1)(n+1)$ Knoten. Jeder Knoten hat 3 Vorgänger (außer am Rand).
% Gittergraph
Editierabstand ist der kürzeste Weg von $(0,0)$ zu $(m,n)$
\subsubsection{Annahmen}
Zwei aufeinanderfolgende Änderungen lohnen sich nicht:
x$\rightarrow$y$\rightarrow$u (Dreiecksungleichung $x$: $\delta_{xy}+\delta_{yu}\geq \delta_{xu}$)\\
Löschen und Einfügen statt Ändern ist in diesem Algorithmus vorgesehen. Falls $\delta_{xy} \leq k_L + k_E$\\
\subsubsection{Speicherreduktion}
auf $\mathcal{O}(m+n)$ durch devide \& conquer.
\begin{enumerate}
\item Zur Berechnung der \underline{Kosten} $f(m,n)$ ist nun $\mathcal{O}(m+n)$ Speicher notwendig: Es genügt, zwei aufeinanderfolgende Spalten im Speicher zu halten.
\item Abstände zum Zielknoten:
\begin{align*}
g(i,j) = \min	\{ g(i+1,j) + k_L, g(i,j+1) + k_E, g(i+1, j+1) + \delta_{a_{i+1},b_{j+1}}\}
\end{align*}
\item Optimallösung =
\begin{align*}
\min\{ f(i,\lfloor \frac{n}{2} \rfloor) , g(i,\lfloor \frac{n}{2} \rfloor) | i = 0..m\}
\end{align*}
% Bild schema der Matrix
\item Es sei $i_0$ das Optimum in 3.\\
Bestimme rekursiv den kürzesten Weg von $(0,0)$ zu $(i_0, \lfloor \frac{n}{2} \rfloor)$ und von $(i_0, \lfloor \frac{n}{2} \rfloor)$ zu $(m,n)$
\end{enumerate}
$\Rightarrow $ Speicherbedarf $\mathcal{O}(m+n)$
\subsection{Laufzeit (inkl. Speicherreduktion)}
Rekursion: $T(m,n) = O(m,n) + T(i_0,  \lfloor \frac{n}{2} \rfloor) + T(m-i_0, n- \lfloor \frac{n}{2} \rfloor)$\\
$n$ sei eine zweier Potenz:
\begin{align*}
T(n,m) &\leq cmn + T(i, \frac{n}{2}) + T(m-i,\frac{n}{2})\\
&\leq cmn + \min\{ T(i,\frac{n}{2}) + T(m-i, \frac{n}{2}) | 0 \leq i \leq m\}
\end{align*}
Behauptung: $T(m,n) \leq c'mn$; Beweis durch Induktion\\
Einsetzen: R.S.\\
\begin{align*}
&cmn + c' i * \frac{n}{2} + c'(m-i)\frac{n}{2}\\
&= cmn + \frac{c'}{2}mn \leq c'mn\\
\end{align*}
wähle $c'=2c$, dann gehts.\\
\section{Gierige Algorithmen}
Optimierungsalgorithmen, die eine Folge von Entscheidungen kurzsichtig treffen und später nicht mehr rückgängig machen.\\
\subsection{Beispiel Rucksack}
Rucksackproblem $g_1,...,g_n G$ mit $w_1, ..., w_n$ \\
\begin{enumerate}
\item Wählt die wertvollsten Gegenstände zuerst, bis Rucksack voll ist.\\
\textbf{NICHT OPTIMAL!}
\begin{tabular}{|c|c|c|c|c|c|}
$g_i$ & 10 & 1 & 1 & \dots & 10\\
\hline
$w_i$ & 100 & 99 & 99 & \dots & \\
\end{tabular}
\item Sortiere Gestenstände $\frac{w_1}{g_1}\geq ... \geq \frac{w_n}{g_n}$ und wähle gierig in dieser Reihenfolge.
\textbf{NICHT OPTIMAL!}
\begin{tabular}{|c|c|c|c|c|c|}
$g_i$ & 8 &	5& 5 & \dots & G = 10\\
\hline
$w_i$ & 9 & 5 & 5 & \dots & \\
\hline
$\frac{w_i}{g_i}$ & $\frac{9}{8}$ & 1 & 1 & \\
\end{tabular}
\end{enumerate}
Bemerkung: Algorithmus 2. ist optimal für das gebrochene Rucksackproblem!\\
Wir sehen, dass Greedy-Algorithmen nicht immer die beste Lösung für gewisse Probleme sind.
\subsection{Umgewichtete Intervallauswahl}
$[a_1,b_1)...[a_n,b_n)$, keine Gewichte $w_i$\\
Wähle eine möglichst große \textbf{Anzahl} von undisjunkten Intervallen aus.\\
\begin{enumerate}
\item Wähle kürzeste Intervalle zuerst. (NICHT OPTIMAL)
\item Intervalle von links nach rechts. (NICHT OPTIMAL)
\item Intervall, das am wenigsten andere überlappt.
\item sortiert nach Endzeitpunkt $b_i$
\end{enumerate}