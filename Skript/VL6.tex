\section{Master Theorem (Fortsetzung) \tiny (Vorlesung 6 am 3.11.)}
\subsection{Beweis Fortsetzung}
\begin{enumerate}
\item[Fall (+)]
\begin{align*}
T(n) & \leq T(\lceil \frac{n}{b} \rceil)+f(n)\\
T(n) &\geq T(\lfloor \frac{n}{b} \rfloor)+f(n)\\
f(n) &= \Theta(n^{\gamma + \epsilon})\\
\gamma &= \log_b a\\
\text{oder: } & \forall n > n_0: \quad a.f(\lceil \frac{n}{b} \rceil) < c.f(n)\\
c<1 \text{ ist eine Konstante }\\
\Rightarrow T(n) &= \Theta(f(n))\\
\end{align*}
\end{enumerate}
\subsection*{Beweis (Induktion)}
untere Schranke $T(n) \geq f(n)$ (aus der Rekursion) $\Rightarrow T(n) = \Omega(f(n))$\\
obere Schranke: Ansatz: $T(n) \leq D.f(n)$\\
Versuch eines Beweises durch Induktion.\\
$n_0$ groß genug machen, dass $\frac{n_0}{b} \leq n_0 -1\Rightarrow \frac{n}{b} \leq n-1 \forall n \geq n_0$\\
$\Rightarrow \lceil \frac{n}{b} \rceil < n$ Induktion kann funktionieren.\\
Induktionsschritt: $n \geq n_0$ für $i < n$ sei $T(i) \leq D.f(i)$ schon bewiesen.\\
\begin{align*}
T(n) & \leq a T(\lceil \frac{n}{b} \rceil) + f(n)\\
& \leq a.D.f(\lceil \frac{n}{b} \rceil) + f(a) \quad \text{ nach I.V.}\\
& \leq D.cf(n) + f(n) \quad \text{Regularitätsbedingung}\\
& \leq D.f(n)\\
\end{align*}
% ! über dem letzen \leq
\begin{align*}
\underbrace{Dc+1 \leq D}_{\text{notwendig}}\\
\leftrightarrow D(1-c) \leftrightarrow D \geq \frac{1}{1-c}\\
\end{align*}
Induktionsbasis: Wähle D groß genug, dass $T(i) \leq Df(i)$ für $i=1,2,...,n_0-1$ gilt.\\
(Voraussetzung: $f(i) > 0$)\\
$D = \max\{\frac{T(1)}{f(1)},\frac{T(2)}{f(2)},\dots,\frac{T(n_0)}{f(n_0)},\frac{1}{1-c}\}$\\
2. Fall: $f(n)=\Theta(n{\gamma + \epsilon}), \epsilon > 0$
\begin{enumerate}
\item[Obere Schranke]
\begin{enumerate}
\item Ersetze $f(n)$ durch $u.n^{\gamma + \epsilon}$\\
Beweise, dass $f(n) = u.n^{\gamma + \epsilon}$ die Regularitätsbedingung erfüllt. (zunächst ohne Aufrunden, weil leichter).\\%f -schlange (tilde über f)
$a.f(\frac{n}{b}) < c.f(n)$
L.S. $= a.u.(\frac{n}{b})^{\gamma + \epsilon} = \frac{a.u n^{\gamma + \epsilon}}{b^\gamma . b^\epsilon}$\\
R.S. $= c.u.n^{\gamma + \epsilon}$
\end{enumerate}
$n_0$ so groß wählen, dass $\frac{(\frac{n}{b}+1)^{\gamma + \epsilon}}{(\frac{n}{b})^{\gamma + \epsilon}}$ nahe genug bei 1 ist, sodass die L.S. immer noch $< cf(n)$ ist.\\
\end{enumerate}
$\Leftarrow (1+\frac{b}{n_0})^{\gamma	+ \epsilon} < b^\epsilon \leftarrow n_0$ groß genug wählen.  
\subsection{Zählen von Fehlständen (Inversion)}
Ein Fehlstand ist ein Paar $a_i > a_j$ mit $i>j$.\\
$(7,3,17,12,16,20) = (a_1,\dots,a_n)$\\% eckige Klammern über 17,..,16; eckige Klammern unter 7,3 und 17,12
\underline{$0\leq \#$Fehlstände $\leq \binom{n}{2}$}
\subsubsection{Divide and Conquer - oder - Warum Mergesort so wichtig ist!}
Fehlstände können zwischen linker und rechter Hälfte leicht bestimmt werden, wenn man die beiden sortierten Listen verschmelzt.\\
$(1 5 6 10)(2 4 7 9)$ 
% siehe Bild
Anzahl der Fehlstände = Anzahl der Fehlstände links + Anzahl der Fehlstände rechts\\
$F((a_1,\dots,a_n)\dots$ Ausgabe: Sortierte Liste$(b_1,\dots,b_n),k$ wobei $k=\#$Fehlstände\\
\begin{lstlisting}[mathescape]
if n=0:return (a_1),0
n'=\lfloor \frac{n}{2} \rfloor; n'' = n-n'
(b_1,...,b_n),F_L = F(a_1,...,a_n')
(c1,...,c_n''),F_R = F(a_{n'+1},...,a_n)
k = F_L + F_R
i = j = 1;
for l = 1,2,...,n
	if (b_i \leq c_j or j = n'' +1) and i \leq n'
		d_l = b_i; k = k + (j-1)
		i ++
	else
		d_l = c_j
		j++
	return (d_1,...,d_n),k
\end{lstlisting}
\subsubsection{Variante}
Länge des Fehlstands ist $j-i (a_i > a_j, j>i)$\\
$p =$ Gesamtlänge alle Fehlstände; wir brauchen zusätzlich zu jeden Element die Position in der ursprünglichen Liste.\\
Eingabe: $a_1,...,a_n$... Ausgabe ist $(b_1,...,b_n),(q1,...,q_n),k,p$ \\
$q_i$ ist die Position von $b_i$ in der Liste $(a_1,...,a_n)$... $(q_i)$ ist eine Permutation von $(1,...,n)$\\
Rekursive Aufrufe....\\
$(b_1,...,b_n),(q1,...,q_n'),F_L,P_L$ = rekursiv
$(c_1,...,c_n),(r1,...,r_n''),F_R,P_R$ = 
\begin{lstlisting}[mathescape]
i = j = 1
for l = 1,...,n
	if i <= n' and (j = n''+1 or b_i \leq c_j)
		d_l = b_i
		s_l = q_i
		k = k + j -1
		// eckige klammer rechts neben die oberen 3 ausdrücken
		p = p + (j-1) (n'-q_i)+T
		// ende
		i++
	else
		d_l = c_j
		s_l = r_j + n'
		// eckige Klammer rechts neben der beiden oberen ausdrücke:
		T = T + r_j
		// ende
		j ++
return (d_1,...,d_n)(s_1,...,s_n),k,p
\end{lstlisting}

\subsubsection{Laufzeit}
Nach Master-Theorem:
\begin{align*}
T(n) = T (\lfloor \frac{n}{2} \lfloor) + T(\rceil \frac{n}{2} \lceil) + \Theta(\underbrace{n}_{n^\gamma (=)})\\
a = 2, b= 2, \gamma = \log_2 2 = 1\\
T(n) = \Theta (n \log n)\\
\end{align*}

Oft teilt das Problem auf, dass man Größen in zwei (ungefähre) gleich große Teile zerlegen möchte, einen Teil mit den kleineren Werten, und einen Teil mit den größeren Werten. \\
Der \textbf{Median} (= $\frac{n}{2}$) - größtes Element ist der ideale Trennungspunkt.
%Bild III