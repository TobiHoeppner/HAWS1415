\section{Geometrie \tiny (Vorlesung 29 am 06.02.)}
\subsection{Konvexe Hülle}
Gegeben sind $n$ Punkte $P_i=(x_i/y_i)$ in der Ebene. Die konvexe Hülle ist die kleinste konvexe Menge, die die gegebenen Punkte enthält.\\
Eine Menge ist konvex, wenn sie mit je zwei Punkte $x,y$ auch die gesamte Verbindungsstrecke enthält.\\
\subsubsection{Einfacher Algorithmus - Einwickeln (gift-wrapping) - Jarvis March}
Beginne bei einem Punkt und berechne jeweils die Strecken zu anderen Punkten. Suche anschließend die Strecke aus, die ganz außen liegt.\\
$\mathcal{O}(n)$ Zeit pro gefundene Ecke. $\mathcal{O}(n^2)$ Zeit insgesamt.
\subsubsection{Besserer Algorithmus - Variante von Grahamscan ohne Winkel.}
$L,R$ sei der linkeste und rechteste Punkt. $L$ und $R$ zerlegen die konvexe Hülle in die untere Hülle und die obere Hülle.\\
\begin{enumerate}
\item[a)] Berechnung der unteren Hülle (obere Hülle analog):\\
Sortieren nach $x$-Koordinate. Annahme:\\
$x_1 < x_2 < ... < x_n$ (alle $x$-Koordinaten verschieden!)\\
$P_1 = L, P_N = R$\\
Inkrementelle Berechnung der unteren Hülle von $P_1,P_2,...,P_k$\\
$k=1,2,...,n$\\
Hinzufügen eines Punktes $P_{k+1}$:\\
untere Hülle von $(P_1,...,P_k)$ liegt auf einem Stapel.\\
while: letzte Kante der untere Hülle (mit mindestens 2 Punkten) macht mit $P_{k+1}$ einen Rechtsknick.\\
do: entferne letzte Ecke und Kante der Unteren Hülle.\\
stapel $s[1],s[2],...,s[l]$\\
$s[1] = P_1, l = 1$\\
for $k=2,3,...,n$\\
while $l \geq 2$ and $s[l-1],s[l],P_k$ macht einen Rechtsknick.\\
do: $l=l-1$\\
$l=l+1,s[l]=p_k$\\
untere Hülle ist $s[1],s[2],...,s[l]$\\
\textbf{Sonderfälle:}
gleiche x-Koordinate.
\begin{enumerate}
\item drehen
\item überlappen
\end{enumerate}
3 Punkte auf einer Geraten oder kein Knick.\\
amortisierte Laufzeit für das Hinzufügen von $P_k$ ist $\mathcal{O}(1)$\\
Potentialfunktion $\Psi = l = $Länge der unteren Hülle.\\
tatsächliche Laufzeit: $\mathcal{O}(1)+\mathcal{O}(\#$Tests in der while-Schleife$)$.\\
$\Delta \Psi = \Psi{vor} - \Psi{nachher} = \#$Tests$-2$\\
amortisiere Laufzeit = tatsächliche - $\Psi^{vor}+\Psi{nach} = 1 +\#$Tests$-\#$Tests$+2=3$\\
$\Psi_0=1, \Psi \geq 1$\\
tatsächliche Laufzeit für untere Hülle $= \mathcal{O}(n)$, Gesamtlaufzeit $= \mathcal{O}(n \log n)$\\
\end{enumerate}
\subsubsection{Orientierungstests für drei Punkte}
$P_1= (x_1,y_1),P_2= (x_2,y_2),P_3= (x_3,y_3)$\\
$D=$
$\begin{matrix}
1 & 1 & 1 \\
x_1 & x_2 & x_3 \\
y_1 & y_2 & y_3 \\
\end{matrix}$
$=$
$\begin{matrix}
1 & 0 & 0\\
x_1 & x_2-x_1 & x_3-x_1\\
y_1 & y_2-y_1 & y_3-y_1\\
\end{matrix}$
$=$
$\begin{matrix}
x_2 -x_1 & x_3-x_1\\
y_2-y_1 & y_3 - y_1\\
\end{matrix}$
$= (x_2-x_1)(y_3-y_1) - (x_3-x_1)(y_2-y_1)$

$D=0 \iff P_1,P_2,P_3$ liegen auf einer Geraden.\\
$D>0 \iff P_1,P_2,P_3$ ist gegen den Uhrzeigersinn orientiert\\
$D>0 \iff P_3$ liegt links von der orientierten Geraden $(P_1,P_2)$\\
$D<0$ analog\\
$\frac{1}{2}|D|=$Fläche des Dreiecks $P_1,P_2,P_3$\\
$n=(P_3-P_1)^\bot$ (um $90^\circ$ nach links gedreht) $= \binom{y_3-y_1}{-(x_3-x_1)}$\\
$h=$ Höhe des Dreiecks $=$ Projektion von $P_2-P_1$ auf die Richtung von $n$.\\
$h=\frac{(n,P_2-P_1)}{||n||}$\\
Fläche = $\frac{\text{Grundfälche}*h}{2} = \frac{||P_3-P_1||*h}{2}=\frac{1}{2}(n,P_2-P_1)=\frac{1}{2}... = \frac{D}{2}$\\
\subsection{Nächstes Paar - Problem (Teile und Herrsche)}
(engl. closest pair)\\
$n$-Punkte, gesucht ist $i,j$ mit kleinstem Abstand $||P_i-P_j||$\\
Teile und Herrsche: Zerlegen in 2 Hälften nach $x$-Koordinaten\\
Rekursiv:\\
$d_L = $kleinster Abstand links\\
$d_R = $kleinster Abstand rechts.\\
$d = \min(d_L,d_R)$ Abstände zwischen linken und rechten Punkten.\\
Es reichen die Punkte in einem Streifen der Breite $2d$ um die Trennungsgerade.\\
Sortiere die Punkte des Steifens nach $y$-Koordinate $\rightarrow$Liste$L$, Berechne den Abstand jedes Punktes zu seinem $7$ Nachbarn in der Liste und merke dir das Minimum $d_{\min}$.\\
Vorverarbeitung: Sortieren nach $x$(erleichtert das rekursive zerlegen) $\mathcal{O}(n \log n)$\\
Sortieren nach y: nach y sortierte Liste wird zerlegt und den Teilproblemen mitgegeben: $\mathcal{O}(n)$\\
Teilliste $L$ kann in $\mathcal{O}(n)$ Zeit aus der sortierten Liste extrahiert werden. \\
$T(n) = 2 T(\frac{n}{2}) + \mathcal{O}(n) \rightarrow T(n) = \mathcal{O}(n \log n)$+Vorverarbitung $\mathcal{O}(n\log n)$