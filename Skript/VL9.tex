\section{Dynamische Programmierung (Fortsetzung)\tiny (Vorlesung 9 am 14.11.)}

\subsection{Gewichtete Intervallauswahl}
Gegeben: $n$ Intervalle $[a_1,b_1),[a_2,b_2),...,[a_n,b_n)$ mit Gewichteten $w1,...,w_n$\\
Gesucht: Disjunkte Intervalle mit dem größten Gesamtgewicht.\\
% Bild I (Intervalle)
Teilprobleme: Betrachte nun Intervalle, die in $(-\infty, x)$ enthalten sind, $x \in \mathbb{R}$. Für $x$ reicht es, die Intervallendpunkte zu betrachten.\\
\subsubsection{Vorverarbeitung (Normalisierung)}
Sortiere alle Intervallendpunkte, ändere die vorkommenden Werte in $1,2,3,...,2n, \Rightarrow x \in \{0,1,2,...,2n\}$. Das Problem wird dadurch nicht verändert.\\
$f(i)=$ größtes Gewicht einer Menge disjunkter Intervalle die in $(- \infty, i)$ enthalten sind.\\
$f(i) = \max \{ f(i-1), \max \{f(a_k)+w_k|$Intervalle $k$ mit $b_k=i\}\}$\\
$f(0)=0$\\
Berechne $f(1),f(2),..,f(n)$ mit der Rekursionsformel. $f(m)$ ist die Optimale Lösung.\\
\subsubsection{Laufzeit}
Jedes Intervall $[a_k,b_k)$ kommt genau $1x$ in der Rekursion auf der rechten Seite vor.\\
\subsubsection{Algorithmus}
\begin{enumerate}
\item[1.($\mathcal{O}(n \log n)$)] Sortieren und Umnummerieren der Endpunkte 
\item[2.($\mathcal{O}(n)$)] Erstelle für $i=1,..,m$ eine Liste $L_i$ der Intervalle $k$ mit $b_k=i$, Initialisiere alle $L_i=\emptyset$\\
\begin{lstlisting}
for k = 1,..n
	L_b_k.append(k)
\end{lstlisting}
\item[3.)] Rekursion:
\begin{lstlisting}
f(i) := max(f(i-1),max(f(a_k)+w_k)|k\in L_i)
\end{lstlisting}
$f$ wird in einem Feld gespeichert.\\
\end{enumerate}
%Bild mit Graphen...

\subsection{Rundreiseproblem (Traveling Salesperson Problem[TSP])}
Gegeben ist ein gerichteter Graph mit Kantengewichten.\\
Gesucht ist ein Kreis, der jeden Knoten genau einmal besucht und geringste Gesamtlänge hat (Hamiltonkreis).\\
Mögliche Lösungen: Startknoten beliebig fixieren.\\
$(n-1)(n-2)(n-3)*...*2*1=(n-1)!$\\
falls der Graph vollständig ist.\\
% Bild mit zwei Teilmengen die je einen Graphen enthalten
Teilprobleme $(T,i) T \subseteq \{1,...,n\}, i \in T, 1 \in T$\\
$f(T,i)=$ der kürzeste Weg von 1 nach $i$ der genau die Knoten in $T$ besucht.\\
\subsubsection{Rekursion}
\begin{align*}
f(T,i) &= \underbrace{\min}_{j \in T-\{i\}, j, i \in E, j \neq 1}(f(T-\{i\},j)c_{ji})&\text{, für } |T| \geq 3\\
f(\{1,i\},i) &= c_{1i} (\text{bzw. }\infty\text{, falls }1i \notin E)\\
\text{OPT}&=\min_{j\neq1, jn \in E}(f(\{1,...,n\},j)+c_{jn})\\
\end{align*}
Wieviele Teilprobleme gibt es?
\begin{align*}
\#\text{Teilprobleme} \leq 2^n.n\\
\end{align*}
$2^n-1$ Teilmengen $T$\\
Zu $T$ gibt es $|T|-1 Teilprobleme (T,i)$ ($\sum (\binom{}n{k})(k-1)$)\\
Jedes Teilproblem benötigt $\mathcal{O}(n)$ Zeit (eigentlich $\mathcal{O}(|T|)$)\\
Insgesamt $\mathcal{O}(2^n n^2)$ Laufzeit\\
Speicher $\mathcal{O}(2^n n)$\\
(exponentiell viel besser als $\mathcal{O}((n-1)!)$)
\subsubsection{1. Möglichkeit}
Tabelle mit $2^{n-1} \times n$ Einträgen. Teilprobleme werden z.B. niht wachsendem $|T|$ gelöst.(Andere Möglichkeit: $T$ als ($n-1$)-stellige Binärzahl darstellen, in nummerischer Reihenfolge lösen.) Wichtig: $T \leq S$ und $T$ vor $S$ lösen.
\subsubsection*{2. Möglichkeit}
Tabellieren (Memoization)\\
Top-down-Berechnung rekursiv nach Bedarf mit Speicher, der schon berechneten Ergebnisse.\\
Initialisieren der Tabelle $M$ auf -1 (Annahme $c_{ij}\geq 0$)\\
\begin{lstlisting}
def f (T,i):
	if M[T,i] != -1: return M[T,i]
	berechne E = f(T,i) nach der Rekursionsgleichung rekursiv.
	M[T,i] = E
	return E
\end{lstlisting}
Man kann sich überlegen, dass genau die Teilprobleme gelöst und gespeichert werden, für die es einen Weg von $i$ nach 1 gibt, der gewanderte Knoten $\{1,...,n\}-T$ als Zwischenknoten besucht.\\
Wenn der Graph wenige Knoten enthält, dann können das viel weniger als $2^nn$ Teilprobleme sein.\\
\subsubsection*{Verwendung einer Hashtabelle für M}
In der Praxis sind RRP mit bis zu 10.000 Ständen bis zur Optimalität lösbar und größere genügend gut approximierbar. Ein Ansatz ist branch-and-bound (Systematisches Durchsuchen von Lösungsbäumen) 