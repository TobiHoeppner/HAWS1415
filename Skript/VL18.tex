\section{Kürzeste Wege (Fortsetzung)\tiny (Vorlesung 18 am 15.12.)}
\subsection{Berechnen aller Paare von kürzesten Wegen}
\begin{enumerate}
\item kürzeste Wege von einem Startpunkt $k$, $n$ mal wiederholen - bspw. Bellman-Ford $n\times \mathcal{O}(mn^2) \leq \mathcal{O}(n^4)$
\item Floyd-Warshall, dynamische Programmierung über Matrizen
\end{enumerate}
Es geht aber auch besser:\\
\subsubsection{Verbesserung}
Definiere beliebige Knotengewichte $u_i, i \in V$\\
Betrachte modifizierte Kosten: $\bar{c_{ij}} = c_{ij} + u_i - u_j$\\ 
% Bild mit Graphen und modifizierten Gewichten
Die kürzesten Wege bezüglich $\bar{c}_{ij}$ sind die gleichen wie die kürzesten Wege bezüglich $c_{ij}$. Der Weg $i_1,i_2, ... , i_k$ berechnet sich wie folgt: $\bar{c}_{i_{1}i_{2}} + \bar{c}_{i_2i_3} +  ... + \bar{c}_{i_{k-1} i_{k}} = (c_{i_1 i_2} + u_{i_1} - u_{i_2}) + (c_{i_2 i_3} + u_{i_2} - u_{i_3}) + ... = (c_{i_1 i_2} + c_{i_2 i_3} + c_{i_{k-1} i_k}) + u_{i_1} - u_{i_k}$ für alle Wege von $i_1$ nach $i_k$ konstant.\\
Idee: finde $u_i$, sodass $\bar{c_{ij}} \geq 0 \quad \forall ij \rightarrow$ Algorithmus von Djikstra $\Rightarrow \mathcal{O}(n \log n +m)$ anwendbar und damit verbessert sich die Laufzeit anstelle von $\mathcal{O}(nm)$\\
Beobachtung: $d_i^*$ sei die Länge des kürzesten Weges von einem festen Startknotens (z.B. $s=1$) zu $i$. Dann wir mit $u_i := d_i^*$ die modifizierten Kosten $\bar{c_{ij}} \geq 0$\\
Beweis: $\bar{c_{ij}} = c_{ij} + u_i + u_j = c_{ij} + d_i^* - d_j^* \geq 0 \Leftrightarrow d_j^* \leq d_i^* + c_{ij}$\\
Jeder Knoten muss von $s$ erreichbar sein. Man kann einen künstlichen Knoten $s$ mit Kanten $s,i \forall i \in V$ und Kosten $c_{si} = 0$\\
\subsubsection{Algorithmus}
\begin{enumerate}
\item Füge künstlichen Startknoten $s$ mit Kanten $c_{si} = 0$ dazu.
\item Bellman-Ford-Moore-Algorithmus mit $s$ als Startpunkt.
\begin{enumerate}
\item Wenn negativer Kreis auftritt, Abbruch.
\item kürzeste Distanzen $d_i^*$
\end{enumerate}
\item Für jeden Startknoten $i$: Alogrithmus von Dijkstra mit Kosten $\bar{c_{ij}}$.
\item Ergebnisdistanzen müssen noch umgerechnet werden.
\item kürzester Weg von $i$ nach $j$... $-u_i + u_j$ addieren.
\end{enumerate}
\subsubsection{Laufzeit}
\begin{enumerate}
\item[2.] $\mathcal{O}(mn)$
\item[3.] $n * \mathcal{O}(m+n*\log n)$
\item[Summe] $\mathcal{O}(mn+n^2\log n)$
\end{enumerate}
\subsection{Floyd-Warshall Algorithmus zur Berechnung aller kürzesten Wege}
\subsubsection{Teilprobleme}
$c_{ij}^k = $ alle Wege von $i$ nach $j$, die als Zwischenknoten nur die Knoten $1,2,...,k$ verwenden dürfen.
\subsubsection{Lösungen}
$c_{ij}^0 = c_{ij} \quad (i\neq j)$
$c_{ii}^0 = c_{ii} $ (Wir legen fest: der leere Weg von $i$ nach $i$ verwendet als Zwischenknoten den Knoten $i$ - das ist im Grunde kein Unterschied mehr zu $c_{ij}^0$)
\subsubsection{Rekrusion}
$(c_{ij}^{k-1})_{i,j \in V} \rightarrow (c_{ij}^k)_{i,j}$ 
Ein Weg in $c_{ij}^k \quad (i,j \neq k)$ hat zwei Möglichkeiten:
\begin{enumerate}
\item Er verwendet den Knoten $k$ überhaupt nicht. $\rightarrow c_{ij}^{k-1}$
\item man kann den Weg aufspalten in 
\begin{enumerate}
\item Anfangsstück von $i$ nach $k$ $\rightarrow c_{ik}^{k-1}$
\item Beliebig viele $\leq 0$ Zwischenstpck von $k$ nach $k$ $\rightarrow c_{kk}^{k-1}$
\item Ein Endstück von $k$ nach $j$ $\rightarrow c_{kj}^{k-1}$
\end{enumerate}
Jedes dieser Stücke verwendet $k$ nicht als Zwischenknoten
\end{enumerate}
$(i,j \neq k) \quad c_{ij}^k = c_{ij}^{k-1} \oplus c_{ik}^{k-1} \otimes [e_1 \oplus c_{kk}^{k-1} \oplus (c_{kk}^{k-1})^2 \oplus (c_{kk}^{k-1})^3 \oplus ...] \otimes c_{kj}^{k-1}$\\
Der Term in den eckigen Klammern ist eine unendliche Summe der folgenden Gestalt:
$e_1 \oplus a \oplus a^2 \oplus a^3 \oplus ... := a^*$, wobei bspw. $a^3$ als $a \oplus a \oplus a$ zu verstehen ist.\\
Wir haben angenommen, dass die Rechengesetze (Distrib. etc.) auch für unendliche Summen gelten, sofern diese Existieren.\\
$i,j \neq k \quad c_{ij}^k = c_{ij}^{k-1} \oplus c_{ik}^{k-1} \oplus (c_{kk}^{k-1})^* \otimes c_{kj}^{k-1}$\\
$j\neq k \quad c_{kj}^k = (c_{kk}^{k-1})^* \otimes c_{kj}^{k-1}$\\
$i\neq k \quad c_{ik}^k = (c_{kk}^{k-1})^* \otimes c_{ik}^{k-1}$\\
$i\neq k \quad c_{kk}^k = (c_{kk}^{k-1})^*$ \\
\subsubsection{Beispiele für $a^*$}
$\oplus = \min$\\
$\otimes = +$\\
$a^* = \min \{0, a, 2a, 3a,... \} = \begin{cases}0, &a\leq 0 \\ \text{undefiniert, sonst} \end{cases}$\\
$\oplus = +$\\
$\otimes = *$\\
$a^* = 1 + a + a^2 + a^3 + ... = \begin{cases} \frac{1}{1-a}, &|a|< 1 \\ \text{undefiniert, sonst} \end{cases}$\\
zweit kürzeste Wege:
$a = (u_1,u_2) \in \mathbb{R}^2, u_1 \leq u_2$\\
$a^* = e_1 \oplus a \oplus a^2 \oplus ... (0,\infty) \oplus (u_1,u_2) \oplus (2u_1, u_1 + u_2) \oplus ...  = \begin{cases} (0,u_1), & u_1 \leq 0 \\ \text{undefiniert, sonst} \end{cases}$\\

\subsubsection{Algorithmus}
Initialisiere $c_{ij}^0 = c_{ij}$
for $k=1,..,n$
\begin{enumerate}
\item Berechne $c_{kk}^k$
\item Zeile $k$: $c_{k_j}^k$ für $j\neq k$
\item Spalte $k$: $c_{ik}^k$ für $i\neq k$
\item Berechne die übrigen Elemente $c_{ij}^k$ für $i,j \neq k$
\end{enumerate}
Ergebnis: $c_{ij}$ in Laufzeit: $\mathcal{O}(n^3)$ mit $\oplus, \otimes, *$-Operation

\subsection{Halbring der formalen Sprachen}
$H = \sigma^* = $Mengen von Wörtern über einem Alphabet $\sigma$\\
$\oplus = \bigcup$\\
$\otimes = \bigodot$ (Konkatenation)\\
Frage: Welche Sprache wird von einem endlichen Automaten akzeptiert.\\
Einen Automaten kann man als Zustandsgraph interpretieren. In diesem Graphen kann man das algebraische Wegeproblem anwenden.\\
Algorithmus von Kleene (ca. 1950er), zum Beweis, dass jede Sprache, die von einem NEA akzeptiert wird, durch einen regulären Ausdruck beschrieben werden kann.\\