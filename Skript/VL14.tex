\section{Kürzeste Wege\tiny (Vorlesung 14 am 01.12.)}

\subsection{Wiederholung: Djikstras Methode}
Eingabe: gerichteter Graph $G=(V,E)$ mit $m$ Kanten und $n$ Knoten und Kantengewichten $c_{uv}$ für $u,v \in E$.
\subsubsection{Algorithmus nach Dijkstra (1960-1961)}
$c_{uv} \geq 0$\\
kürzester Weg von einem Startknoten zu allen anderen Knoten.\\
\subsubsection*{Laufzeiten}
Laufzeiten optimieren sich hauptsächlich mit der verwendeten Speicherstruktur.\\
\begin{tabular}{|c|c|}
\hline
\textbf{Methode} & \textbf{Laufzeit}\\
\hline
\hline
primitiv & $\mathcal{O}(n^2)$\\
\hline
Halde (heap) als Prioritätswarteschlange & $\mathcal{O}((m+n)\log	 n)$\\
\hline
Fibonacci-Halde & $\mathcal{O}(m + n \log n)$\\
\hline
\end{tabular}
\subsection{Algorithmus von Bellman / Ford}
$c_{uv} \in \mathbb{R}$ beliebig.\\
ein Startknoten.\\
Laufzeit $\mathcal{O}(nm)$\\
\subsubsection*{Algorithmus}
$d_i^* = $ kürzeste Weglänge von Startknoten $s$ zu Knoten $i$ mit \underline{höchstens $k$ Knoten}.\\
$k=0,1,2,...$\\
\subsubsection{Rekursion}
$d_j^k= \min	\{ d_i^{k-1} + c_{ij} \; | \; i \in V, ij \in E\} \cup \{d_j^{k-1}\}$\\
\subsubsection{Anfangswerte}
$d_j^{(0)} = \begin{cases} 0,\quad j=s \\ \infty, \quad \text{sonst}\end{cases}$\\
\subsubsection{Beispiel}
% siehe Bild (julian)
\begin{tabular}{c|ccccc}
j,k & 0 & 1 & 2 & 3 & 4\\
\hline
1 & $0$ &$0$&$0$&$0$&$0$\\
2 & $+\infty$&$-1$&$-1$&$-1$&$-1$\\
3 & $+\infty$&$6$&$1$&$1$&$1$\\
4 & $+\infty$&$5$&$3$&$-2$&$-2$\\
\end{tabular}
\subsubsection{Bemerkungen}
Wenn der Graph keine negativen Kreise enthält, dann besucht ein kürzester Weg keinen Knoten mehrfach und hat somit $n-1$ Kanten.\\
$\Rightarrow d_j^{n-1}$ sind die kürzesten Weglängen.\\
Vektor $d^{(k)} = \begin{pmatrix} d_1^{(k)} \\ \vdots \\d_n^{(k)} \end{pmatrix}$, Rekursion hat die Gestalt $d^{(k)} = F(d^{(k-1)}), d^{(k)} \leq d^{(k-1)}$\\
Sobald $d^{(k-1)} = d^{(k)}$ ist, kann man abbrechen:\\
 $F(d^{(k-1)}) = F(d^{(k)}) \Rightarrow d^{(k)} = d^{(k+1)} \Rightarrow d^{(k+2)} = d^{(k+3)} = ...$\\
 Wenn es einen negativen Kreis gibt, dann kann die Weglänge beliebig klein werden, indem man diesen Kreis beliebig oft durchläuft.\\
 (Sofern der Kreis vom Startknoten erreichbar ist.)\\
 Falls $d^{(n-1)} \neq d^{(n)}$, dann muss es einen negativen Kreis geben. $\rightarrow$ \lstinline!ABBRUCH!.\\
 \textbf{Satz:} Der Algorithmus von Bellmann-Ford bestimmt in $\mathcal{O}(nm)$ Zeit die kürzesten Wege von einem Startknoten zu allen anderen Knoten oder er stellt fest, dass der Graph einen negativen Kreis enthält.\\
\subsubsection{Implementierung der Rekursion}
\begin{enumerate}
\item[a)] wie geschrieben \lstinline!for j = 1,...,n (for! alle eingehenden Kanten \lstinline!)!
\item[b)] "Vorwärtsrechnung" \lstinline!for j = 1,...,n:!$d_j^{(k)} := d_j^{(k-1)}$\\
$\quad$ \lstinline!for i = 1,...,n: (for! \\$\quad\quad$alle ausgehenden Kanten$(i,j)$:\\$\quad\quad\quad$\lstinline!( if! $d_i^{(k-1)}+c_{ij} \leq d_j^{(k)}$ \lstinline!then! $d_j^{(k)} := d_i^{k-1}+c_{ij}$\lstinline!))!
\end{enumerate}
\subsubsection{Verbesserungsmöglichkeit}
$d_i^{(k-1)}$ und $d_i^{(k)}$ nicht unterscheiden, sondern nur 1 Variable verwenden $d_i$\\
\subsubsection{Rekursion neu}
\begin{lstlisting}
for i = 1,...,n:
	# erforschen
	for alle Knoten (i,j):
		if d_i +c_{ij} < d_j then
			d_j = d_i + c_{ij}
\end{lstlisting}
$d_i^{(k-1)} \leq d_i^{(k)} \leq d_i^{(k+1)}$\\
Per Induktion ergibt sich:\\
Nach $k$ Iterationen ist $d_i^{(\text{NEU})} \leq d_i^{(k)}$\\
Wenn es keine neg. Kreise gibt, dann ist $d_i^{(\text{NEU})}$ immer $\leq d_i^{(n-1)}$ (tatsächlich kürzester Weg)!\\
Jedes $d_i$, dass im Algorithmus ausgerechnet wird, ist die Länge eines Weges von $s$ nach $i$.\\
Nach $k$ Schritten gilt:\\
$d_i^{(k)} \leq d^{(\text{NEU})} \leq d_i^{(n-1)}$\\
$k=n-1: \rightarrow d_i^{(\text{NEU})} = d_i^{(n-1)}$\\
Die Knoten $i$, deren Wert sich seit der letzten (Erforschung, Behandlung) geändert haben, speichert man in einer Warteschlange $Q$. (Falls sich der Wert nicht geändert hat, wäre es sinnlos ihn noch einmal zu erforschen).\\
\begin{lstlisting}[mathescape]
erforsche(i):
	for all Kante(i,j):
		if d_i + c_{ij} < d_j then
			d_j = d_i + c_{ij}:
			if j \notin Q then:
				O.insert(j)
\end{lstlisting}
\subsubsection{Verbesserter Algorithmus}
\begin{lstlisting}[mathescape]
d_s = 0, d_i = \infty für i \neq s
Q=(s)
while Q \neq \emptyset:
	Lösche das erste Element i aus Q
	erforsche(i)
\end{lstlisting}
\subsection{Algorithmus von Bellman, Ford, Moore}
Wir unterteilen den Algorithmus in Phasen.
Phase $0$ endet nach der Initialisierung $Q:=(s)$.
Phase $k$ beginnt wenn Phase $k-1$ endet und dauert, bis die Knoten in $Q$, die zu Beginn der Phase in $Q$ sind, erforscht sind.\\
Am Ende von Phase $k$ gilt $d_i \leq d_i^{(k)}$\\
\subsubsection*{nach Induktion}
Am Beginn der Phase gilt $d_i \leq d_i^{(k-1)}$\\
Wir stellen uns vor, dass wir jetzt alle Knoten erforschen.\\
1 Rekursion "neu" $\begin{cases} \text{zuerst die Knoten, die} \\\text{nicht in $Q$ sind}\\\text{dann die Knoten in }Q\end{cases}$\\
Effekt: $d_i\leq d_j^{(k)}$ danach.\\
Um im Falle eines neg. Kreises abbrechen zu können, muss man 
\begin{enumerate}
\item[a)] nach $n^2$ Erforschungen abbrechen
\item[b)] oder die Phasen mitzählen und die Phasengrenzen in $Q$ markieren:\\
\begin{lstlisting}

$Q = \{s, _M\}$\\
Phase := 0\\
while $Q \neq \emptyset$:
	if erstes Elemente = _M:
		entferne _M und füge es am Ende ein.
		Phase = Phase +1
		if Phase > n then Abbruch.
	else:
		lösche erstes Element i und erforsche(i)
\end{lstlisting}
\end{enumerate}
