\section{ Komplexitätstheorie \tiny (Vorlesung 24 am 19.01.)}
\subsection{Stark-NP-vollständig}
Ein Problem heißt stark-NP-vollständig (stark-NP-schwer), wenn es auch dann noch NP-vollst. (-schwer) ist, wenn die Eingabe vollkommenden Zahlen unär kodiert werden: Kodelänge für eine Zahl $x \in \mathbb{N}$ ist $x+1$. (Bspw. $9 = 1111111110$)\\
Algorithmen mit dieser Eingabekodierung heißen pseudopolynomielle Algorithmen.\\
\subsection{Rundreiseproblem ist stark-NP-vollständig}
Beweis:
\begin{enumerate}
\item $\in$ NP
\item HAMMILTONKREIS $\leq$ RUNDREISEPROBLEM (unär)\\
Diese Reduktion ist auch bei unärer Eingabe ein polynomielle Reduktion: die auftretenden Zahlen sind 1,2,n.
\end{enumerate}
3-SAT $<_p$ Teilmenge
\subsection{coNP}
$= \{ A \subseteq \Sigma^* | \Sigma^* - A \in NP\}$\\
Probleme in NP, mit vertauschter Antwort JA $\leftrightarrow$ NEIN\\
Beispiel: Primzahltest:
Eingabe: $n \in \mathbb{N}$
Frage: Ist $n$ eine Primzahl?
Dieses Problem ist in coNP. Zertifikat: zwei Faktoren $p,q > 1$mit Produkt $p*q = n$\\
Das Problem ist sogar in P!.\\
coNP-vollständig , coNP-schwer, analog zu NP.\\
\subsection{PSPACE}
Probleme die in polynomieller Speicherplatz lösbar.\\
PSPACE-vollst. - PSPACE-schwer\\
Landkarte der Komplexitätsklassen:\\
% BILD1
\subsection{Turing-Reduzierbarkeit}
Ein Berechnungsporblem $A$ ist NP-schwer, wenn man mit seiner Hilfe (man darf das Problem A beliebig oft als Unterprogramm aufrufen) jedes Problem in NP in polynomieller Zeit löse könnte.\\
(bisher) $\leq_p$ - komplette Reduzierbarkeit: die Eingabe von B in eine Eingabe von A umrechnen und darf dann den Algorithmus von A einmal verwenden und muss die Antwort (JA / NEIN) unverändert übernehmen.\\
$x \in B \Leftrightarrow f(x) \leftarrow A$
\section{Approximationsalgorithmen}
\subsection{amortisierte Analyse}
Beispiel isotone Regression:
$Q = \emptyset$\\
wiederhole n-mal:\\
Q.insert(...)\\
while bla\\
...\\
Q.deletemax()\\
Ein Schleifendurchlauf kann bis zu $\mathcal{O}(n)$ Zeit dauern. In Summe als locker in $\mathcal{O}(n^2)$. Aber trotzdem ist die Gesamtlaufzeit nur in $\mathcal{O}(n)$, weil aus $Q$ nicht öfters gelöscht, als eingefügt werden. Die amortisierte Laufzeit eines Schleifendurchlaufes ist $\mathcal{O}(1)$.\\
Definition: Datenstrukturen mit Operationen $A,B,C,...$\\
Wir sagen, die Operationen haben amrotisierte Laufzeit $T_A(n),T_B(n),T_C(n),...$ bei Eingabe der Größe $n$, wenn jede Folge von $n_A$ Operationen $A$, $n_B$ Operationen B,... auf einer leeren Datenstruktur beginnend mit: $\mathcal{O}(n_A*T_A(n)+n_B*T_B(n)+n_C*T_C(n)+..)$ Zeit ausgeführt werden kann.\\
\subsubsection{Bsp: Fibonacci-Heap}
Fib-Halden sind Priotritätswarteschlange, die die Operationen:
\begin{enumerate}
\item insert
\item findmin
\item merge
\item decreasekey
\end{enumerate}
in $\mathcal{O}(1)$ amortisierter Zeit und 
\begin{enumerate}
\item deletemin
\end{enumerate}
in $\mathcal{O}(\log n)$ amortisierter Zeit ausführen können.\\
Anwendung: Algorithmus von Dijkstra mit $n$ Knoten und $m$ Kanten.\\
\begin{enumerate}
\item $n$ insert $\mathcal{O}(1)$
\item $m$ decreasekey $\mathcal{O}(1)$
\item $n$ deletemin $\mathcal{O}(\log	n)$
\end{enumerate}
$\rightarrow \mathcal{O}(n \log n + m)$\\
\subsection{Binomialbäume}
$B_0$ = Wurzel\\
Induktive Defintion eines Binomialbaumes $B_i$ vom Rang $i$ (=Höhe $i$):\\
$B_i$ erhält man, indem man an die Wurzel von $B_{i-1}$ eine weitere Kopievon $B_{i-1}$ als zusätzliches letztes Kind der Wurzel anhängt.\\
$B_i$ hat $2^i$ Kinder, und zwar $\binom{i}{k}$ Kinder auf Tiefe $k$.\\
Eine Binomialhalde ist eine Liste von Binomialbäumen mit verschiedenen Rang, wobei die Kuchen in jedem Baum die Haldeneingeschaft haben. Schlüssel eines Knotens $\geq$ Schlüssel des Elternknotens.\\
Beispiel: Binomialhalde mit 11 Knoten von $n=11=8+2+1=1011_2$\\
Struktur ist durch die Knotenanzahl $n$ eindeutig vorgegeben. Es gibt eine Binomialhalde $\leq \log_2 n$ Binomialbäume.\\
... findmin $\mathcal{O}(\log n)$ Zeit (Durchsuchen der Wurzeln).\\
\subsubsection{merge}
Vereinigung von zwei Binomialhalden.\\
Zwei $B_i$s können in Konstaner Zeit zu einem $B_{i+1}$ kombiniert werden:\\
Vergleich der Wurzeln: größere Wurzel wird als Kind an die kleinere gehängt.\\
Algorithmus: verschmelze die beiden Listen von Binomialbäumen zu einer Liste, nach Rang sortiert.\\
for $i = 0,1,2,... \lfloor \log n \rfloor$\\
while $\exists$ zwei Bäume $B_i$: vereinige sie zu einem Baum $B_{i+1}$
$B_i$ = Knoten mit Unterbaum $B_{i-1}$\\
Laufzeit: $\mathcal{O}(\log n)$\\
\subsubsection{insert}
merge mit einer neuen Binomialhalde mit +1 Knoten: $\mathcal{O}(\log n)$
\subsubsection{deletemin}
finde die Wurzel mit dem kleinsten Schlüssel. lösche sie, vereinige die Knoten (die selbst eine Binomialhalde bilden) mit den übrigen Wuzeln (merge) ... $\mathcal{O}(\log n)$ Zeit.