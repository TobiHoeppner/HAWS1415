\documentclass[ngerman,a4paper]{report}
\usepackage[ngerman]{babel}
\usepackage[T1]{fontenc}
\usepackage[utf8]{inputenc}
\usepackage{MyriadPro}
\usepackage[scaled]{beramono}
\newcommand\Small{\fontsize{10.5}{10.5}\selectfont}
\newcommand*\LSTfont{\Small\ttfamily\SetTracking{encoding=*}{-20}\lsstyle}
\usepackage{microtype}
\usepackage{geometry}
\geometry{verbose,tmargin=3cm,bmargin=3cm,lmargin=3cm,rmargin=3cm}
\usepackage{listings}
\usepackage{stmaryrd}
\usepackage{paralist}
\usepackage{array}
\usepackage{color}
\usepackage{graphicx}
\usepackage{caption}
\usepackage{url}
\usepackage{amsmath}
\usepackage{amssymb}
\usepackage{accents}
\usepackage{tikz}
\usetikzlibrary{arrows}

%tikz
\tikzset{
  treenode/.style = {align=center, inner sep=0pt, text centered,
    font=\sffamily},
  arn_n/.style = {treenode, circle, white, font=\sffamily\bfseries, draw=black,
    fill=black, text width=3.5em},% arbre rouge noir, noeud noir
  arn_r/.style = {treenode, red, 
    text width=1.5em, very thick},% arbre rouge noir, noeud rouge
  arn_x/.style = {treenode, rectangle, draw=black,
    minimum width=0.5em, minimum height=0.5em}% arbre rouge noir, nil
}

% Code Listing Style
\definecolor{darkblue}{rgb}{0,0,.6}
\definecolor{darkgreen}{rgb}{0,0.5,0}
\definecolor{darkred}{rgb}{0.5,0,0}
\lstset{%
	language=Python, 
	basicstyle=\LSTfont,
	commentstyle=\color{darkgreen}, 
	keywordstyle=\color{darkblue}\bfseries, 
	breaklines=true,
	tabsize=2,
	xleftmargin=\fboxsep,
	xrightmargin=-\fboxsep,
	numbers=left,
	numberstyle=\tiny\color{gray},
	stepnumber=1,
	numbersep=5pt,
	frame=bt,
	stringstyle=\color{darkred},
	showstringspaces=false,
	rulecolor= \color{gray},
	emph=[1]%
	{%  
	    then, not, for, return%
	},
	emphstyle=[1]{\color{darkblue}\bfseries},
	emph=[2]%
	{%  Datatypes
	    %
	},
	emphstyle=[2]{\color{darkblue}\bfseries},
	emph=[3]%
	{%  
	    %
	},
	emphstyle=[3]{\color{darkred}\bfseries},
	literate=%
	{Ö}{{\"O}}1
	{Ä}{{\"A}}1
	{Ü}{{\"U}}1
	{ß}{{\ss}}2
	{ü}{{\"u}}1
	{ä}{{\"a}}1
	{ö}{{\"o}}1
}
\providecommand{\tabularnewline}{\\}

\usepackage{fancyhdr}
\pagestyle{fancy}
\usepackage{lastpage}
\makeatletter

\lhead{\textbf{\@title} \\ \@author}
\chead{}
\rhead{\@date \\ \thepage \ von \pageref{LastPage} }
\cfoot{}

\renewcommand{\maketitle}{}
\newcommand{\utilde}[1]{\underaccent{\tilde}{#1}}
\renewcommand{\familydefault}{\sfdefault}
 
\author{Tobias Höppner}
\title{HA - Beton-Übung 15.}
\date{Wintersemester 2014/2015}

\begin{document} 
\maketitle 

\subsubsection*{82. Fluten eines Tagbaus, digitale Landschaft, 10 Punkte}

TODO: $i,j$-Werte? Reicht es sich jeweils nur $\min$ für jede Menge zu merken?\\

Gegeben: Sei $S_r$ die Menge der Seen und $I_r$ die Menge der Inseln ($r \in \mathbb{N}$). Die Höhe des Grundwasserspiegels wird durch $h$ repräsentiert.\\
Gesucht: Alle Bereiche (Seen) die ab einer Höhe $h$ überflutet werden.\\
Bei der Vorverarbeitung werden alle Felder $a[i,j]$ aus der Landschaft $L = S_0 \cup ... \cup S_r \cup I_0 \cup ... \cup I_r$ mit $h=0$ entsprechend zu Seen und Inseln zusammengefasst. Ein See $S$ (analog Insel $I$) ist eine Menge von Feldern, die direkt benachbart (entweder $a[i+1,j]$ oder $a[i,j+1]$, aber nicht $a[i+1,j+1]$) sind. Der Repräsentant der jeweiligen Felder ist die Höhe $h_{a_{i,j}}$. Zu jeder Menge $S$ oder $I$ werden die Koordinaten $i,j$ gespeichert, sodass schneller bestimmt werden kann, ob ein Feld $a$ direkter Nachbar von dieser Menge ist.\\

\begin{enumerate}
\item[FIND(h)] liefert das erste Feld $a[i,j] <h$ in $L$.\\
\item[UNION(a, S)] entfernt $a$ aus der Ursprungsmenge und fügt es zur neuen Menge $S$ hinzu. Die richtige Menge $S$ ist die Menge, wo $a$ ein direkter Nachbar wird. Gibt es zwei Mengen, wo $a$ direkter Nachbar wird, müssen alle Knoten aus der kleineren Menge in die größere überführt werden. (einfacher in Baumstruktur: $a$ wird neue Wurzel).
\end{enumerate}
\subsubsection*{Algorithmus}
Finde alle Felder $<h$ und füge sie der richtigen Menge hinzu.\\

\subsubsection*{Laufzeit}
Sei $n=|L|$.
\begin{enumerate}
\item[FIND(h)] $\mathcal{O}(n)$
\item[UNION(a,S)] $\mathcal{O}(1)$
\end{enumerate}

\subsubsection*{Die Ackermannfunktion, 10 Punkte}
\begin{enumerate}
\item[a)] Funktionen
\begin{enumerate}
\item[$A_1(n)$]$=\begin{cases}3,& n=1\\ A_{0}(A_1(n-1)),&n>1 \end{cases}$
\item[$A_2(n)$]$=\begin{cases}2,& n=1\\ A_{1}(A_2(n-1)),&n>1 \end{cases}$
\item[$A_3(n)$]$=A_2(A_3(n-1))$
\item[$A_4(n)$]$=A_3(A_4(n-1))$
\end{enumerate}
\item[b)] Werte\\
\begin{tabular}{|c|c|c|}
\hline
$i$ & $A_i(1)$& $A_i(2)$\\
\hline
\hline
$0$ &	$2$	&	$3$\\
\hline
$1$ &	$3$	&	$4$\\
\hline
$2$ &	$$	&	$$\\
\hline
$3$ &	$$	&	$$\\
\hline
$4$ &	$$	&	$$\\
\hline

\end{tabular}
\end{enumerate}

\end{document}
