\documentclass[ngerman,a4paper]{report}
\usepackage[ngerman]{babel}
\usepackage[T1]{fontenc}
\usepackage[utf8]{inputenc}
%\usepackage{MyriadPro}
\usepackage[scaled]{beramono}
\newcommand\Small{\fontsize{10.5}{10.5}\selectfont}
\newcommand*\LSTfont{\Small\ttfamily\SetTracking{encoding=*}{-20}\lsstyle}
\usepackage{microtype}
\usepackage{geometry}
\geometry{verbose,tmargin=3cm,bmargin=3cm,lmargin=3cm,rmargin=3cm}
\usepackage{listings}
\usepackage{stmaryrd}
\usepackage{paralist}
\usepackage{array}
\usepackage{color}
\usepackage{graphicx}
\usepackage{caption}
\usepackage{url}
\usepackage{amsmath}
\usepackage{amsfonts}
\usepackage{accents}
\usepackage{tikz}
\usetikzlibrary{plotmarks}

% Code Listing Style
\definecolor{darkblue}{rgb}{0,0,.6}
\definecolor{darkgreen}{rgb}{0,0.5,0}
\definecolor{darkred}{rgb}{0.5,0,0}
\lstset{%
	language=[x86masm]Assembler, 
	basicstyle=\LSTfont,
	commentstyle=\color{darkgreen}, 
	keywordstyle=\color{darkblue}\bfseries, 
	breaklines=true,
	tabsize=2,
	xleftmargin=\fboxsep,
	xrightmargin=-\fboxsep,
	numbers=left,
	numberstyle=\tiny\color{gray},
	stepnumber=1,
	numbersep=5pt,
	frame=bt,
	stringstyle=\color{darkred},
	showstringspaces=false,
	rulecolor= \color{gray},
	emph=[1]%
	{%  
	    HALT%
	},
	emphstyle=[1]{\color{darkblue}\bfseries},
	emph=[2]%
	{%  Datatypes
	    %
	},
	emphstyle=[2]{\color{darkblue}\bfseries},
	emph=[3]%
	{%  
	    MAX, MIN, TRUE, FALSE, skip%
	},
	emphstyle=[3]{\color{darkred}\bfseries},
	literate=%
	{Ö}{{\"O}}1
	{Ä}{{\"A}}1
	{Ü}{{\"U}}1
	{ß}{{\ss}}2
	{ü}{{\"u}}1
	{ä}{{\"a}}1
	{ö}{{\"o}}1
}
\providecommand{\tabularnewline}{\\}

\usepackage{fancyhdr}
\pagestyle{fancy}
\usepackage{lastpage}
\makeatletter

\lhead{\textbf{\@title} \\ \@author}
\chead{}
\rhead{\@date \\ \thepage \ von \pageref{LastPage} }
\cfoot{}

\renewcommand{\maketitle}{}
\newcommand{\utilde}[1]{\underaccent{\tilde}{#1}}
\renewcommand{\familydefault}{\sfdefault}
 
\author{Tobias Höppner}
\title{HA - Übung 01. }
\date{Wintersemester 2014/2015}

\begin{document} 
\maketitle 
\section*{Aufgabe 2}
\begin{align*}
A \quad	f(n)\\
B \quad 	g(n)\\
g(n) = O (f(n)\log n)\\
\end{align*}
\begin{enumerate}
\item[\textbf{a)}] Beispiele \\
	\begin{enumerate}
	\item[\textbf{i)}] A schneller B
		\begin{align*}
		f(n) &= n^2\\
		g(n) &= n \log n\\
		n \log n = O (n^2 \log n)
		\end{align*}
	\item[\textbf{ii)}] B schneller A
		\begin{align*}
		f(n) = n\\
		g(n) = n \log n\\
		n \log n = O(n^2 \log n)
		\end{align*}
	\end{enumerate}
\item[\textbf{b)}]
	\begin{align*}
	g(n) &= \Omega (f(n) \log n)
	\end{align*}
	$\Rightarrow$ \textbf{i)} gilt laut Definition.\\
\item[\textbf{c)}] ja, $x$ Eingabe und $T(x)$ Laufzeit\\
	\begin{align*}
	T(n) &= \max T(x)\\
	|x| &= n \quad \text{Laufzeit im schlimmsten Fall}\\
	T(n) = \Theta(n^2)\\
	\end{align*}
	$\Rightarrow$ Bubblesort!\\
	$x_i$ Folge von $i$ sortierten Elementen $\rightarrow T(n) = \Theta (n^2)$\\
	$T(x_i) = O(i)$\\
	\begin{align*}
	T(n) &= \Omega (n^2)\\
	y_n = \text{Permutation} &= \{1, 2, 3, ..., n, n-1, n-2, ..., \underline{1}\}\\
	T(y_n) &= \Omega(n^2)\\
	\end{align*}
	Für alle Funktionen f,g in der Vorlesung gilt entweder $f = O(g)$ oder $g = O(f)$, aber: Es gibt Funktionen, für die wir es nicht machen können.\\
\item[\textbf{d)}] A hat Laufzeit $\Theta (n^2)$, kann es sein, dass $\forall x |x|=n = T(x) = O(n)$?\\
	$\Rightarrow$ Nein. Die Laufzeit im schlimmsten Fall von $\Omega (n^2)$ bedeutet, dass es für jedes $n$ eine Eingabe der Länge $n$ gibt, bei der der Algorithmus $n^2$ Zeit benötigt.\\
	A hat Laufzeit $O(n^2)$, kann es sein, dass $\forall x |x|=n = T(x) = O(n)$?
	$\Rightarrow$ ja.\\
	A hat Laufzeit $\Omega (n^2)$, kann es sein, dass $\forall x |x|=n = T(x) = O(n)$?
	$\Rightarrow$ nein.\\
\item[\textbf{e)}]
	\begin{enumerate}
	\item[a)] $3^n = O(2^n)$ Wahr oder Falsch? $\Rightarrow$ Falsch!\\
	$4^n = 2^n * 2^n$, wobei $2^n \underbrace{=}_\text{unmöglich!} O(2^n)$\\
	\item[b)] $\underbrace{\log 3^n}_{n \log 3} = \underbrace{O(\log 2^n)}_{n \log 2} \Rightarrow $ Wahr!\\
	\item[c)] $3^n = \Omega(2^n) \Rightarrow 3^n > 2^n$\\
	\item[d)] $log 3^n = \Omega (log 2^n) \Rightarrow \log 3^n > log 2^n$\\ 
	\end{enumerate}
\item[\textbf{f)}] $4^n + n^2 \simeq 2^{2n} \geq 2^n \geq 2^{\sqrt{n}} \geq n^2 \log_2(n^2) \simeq n^2 \log n \geq n \log 2^n \simeq 5n^2 + n \log n \simeq n^2 + n-1 \geq (\log n)^2 \geq \log_3 n \simeq \log_2 n$\\ 
\end{enumerate}

\textbf{Einschub: }Laufzeitvergleich\\
\begin{align*}
5n^2 + n log n &= \Theta (n^2 + n -1)\\
\\
\text{z.z.:} \quad \quad O(n^2 +n-1)\\
&= \{f | \exists n_0 \in \mathbb{N}, c \in \mathbb{R}^+ \forall n > n_0 \quad f(n)\leq c(n^2)+n-1\}\\
5n^2 + n* \log n &< 100(n^2 + n-1)\\
5 n^2 &< 10n^2\\
n \log n &< n^2 \quad \forall n\\
\end{align*}

\section{RAM}
Ziel ist es eine RAM zu bauen, die eine RAM bekommt und diese dann um 10 Plätze nach unten verschiebt.\\
\textbf{Idee:} Von hinten nach vorne Verschieben. Die Programmlänge ist in \lstinline!R0! gegeben. Es braucht dann zwei freie Register.\\
Zwei freie Register bekommt man so:\\
\begin{lstlisting}[mathescape]
R0 = R0 + 12 	; R_0 = n + 12
(R0) = R2 		; R_{n+12} = x_2
R0 = R0 -1 		; R_0 = n+11
(R0) = R1 		; R_{n+11} = x_1
\end{lstlisting}
Verschieben des Programms (Umschaufeln)
\begin{lstlisting}
R1 = R0 - 11
R2 = R0 -1
loop:
	GZ R1, end
	(R2) = (R1)
	R2 = R2 -1
	R1 = R1-1
	GOTO loop
\end{lstlisting}
Aufräumen...
\begin{lstlisting}
end:
	R11 = R0
	R0 = R0 + 1
	R12 =  (R0)
	R 0 =  R0 - 12
	HALT
\end{lstlisting}
\section*{EKM}
Einheitskostenmaß, jede Operation kostet $1$.\\Schritt:
\begin{enumerate}
\item $8$ bzw. Konstant
\item $4 + \underbrace{8}_\text{Kosten pro Schleife} * \underbrace{n}_\text{Schleifendurchläufe}$
\item Konstant
\end{enumerate}
$\Rightarrow \Theta(n)$\\

\section*{LKM}
Logarithmisches Kostenmaß. Zeilenweise ansehen...\\
1. $l(n) + c$\\
2. $l(n) + l(x_1)$\\
3. $l(n) + c$\\
4. $l(n) + l(x_2)$\\
5. $l(n) + c$\\
6. $l(n) + c$\\
8 - 12. $l(n)*5n$ für Zeilen 8, 10, 11 $+ l(x_3)+...+l(x_n)$ fürs kopieren\\
13 - 17. $c+l(n)+l(x_0) + l(x_2)$\\
$\Rightarrow O(n*l(n) + \sum n _ i = 1 (l(x_i)))$

\end{document}
