\documentclass[ngerman,a4paper]{report}
\usepackage[ngerman]{babel}
\usepackage[T1]{fontenc}
\usepackage[utf8]{inputenc}
%\usepackage{MyriadPro}
\usepackage[scaled]{beramono}
\newcommand\Small{\fontsize{10.5}{10.5}\selectfont}
\newcommand*\LSTfont{\Small\ttfamily\SetTracking{encoding=*}{-20}\lsstyle}
\usepackage{microtype}
\usepackage{geometry}
\geometry{verbose,tmargin=3cm,bmargin=3cm,lmargin=3cm,rmargin=3cm}
\usepackage{listings}
\usepackage{stmaryrd}
\usepackage{paralist}
\usepackage{array}
\usepackage{color}
\usepackage{graphicx}
\usepackage{caption}
\usepackage{url}
\usepackage{amsmath}
\usepackage{accents}
\usepackage{tikz}
\usetikzlibrary{plotmarks}

% Code Listing Style
\definecolor{darkblue}{rgb}{0,0,.6}
\definecolor{darkgreen}{rgb}{0,0.5,0}
\definecolor{darkred}{rgb}{0.5,0,0}
\lstset{%
	language=[x86masm]Assembler, 
	basicstyle=\LSTfont,
	commentstyle=\color{darkgreen}, 
	keywordstyle=\color{darkblue}\bfseries, 
	breaklines=true,
	tabsize=2,
	xleftmargin=\fboxsep,
	xrightmargin=-\fboxsep,
	numbers=left,
	numberstyle=\tiny\color{gray},
	stepnumber=1,
	numbersep=5pt,
	frame=bt,
	stringstyle=\color{darkred},
	showstringspaces=false,
	rulecolor= \color{gray},
	emph=[1]%
	{%  
	    then, not, for, return%
	},
	emphstyle=[1]{\color{darkblue}\bfseries},
	emph=[2]%
	{%  Datatypes
	    %
	},
	emphstyle=[2]{\color{darkblue}\bfseries},
	emph=[3]%
	{%  
	    %
	},
	emphstyle=[3]{\color{darkred}\bfseries},
	literate=%
	{Ö}{{\"O}}1
	{Ä}{{\"A}}1
	{Ü}{{\"U}}1
	{ß}{{\ss}}2
	{ü}{{\"u}}1
	{ä}{{\"a}}1
	{ö}{{\"o}}1
}
\providecommand{\tabularnewline}{\\}

\usepackage{fancyhdr}
\pagestyle{fancy}
\usepackage{lastpage}
\makeatletter

\lhead{\textbf{\@title} \\ \@author}
\chead{}
\rhead{\@date \\ \thepage \ von \pageref{LastPage} }
\cfoot{}

\renewcommand{\maketitle}{}
\newcommand{\utilde}[1]{\underaccent{\tilde}{#1}}
\renewcommand{\familydefault}{\sfdefault}
 
\author{Tobias Höppner, Marian Sigler}
\title{HA - Übung 03. }
\date{Wintersemester 2014/2015}

\begin{document} 
\maketitle 
\section*{18. Rekursion}
Siehe handschriftliche Aufzeichnungen (Seite 3) von Marian. Danke!

\subsection*{Musterlösung aus dem Tutorium 10.11.2014}

\begin{align*}
T(n) &= T(\lfloor \frac{n}{2} \rfloor) + T(\lceil \frac{n}{2} \rceil) +2\\
\end{align*}

\begin{enumerate}
\item[a)] $n =3 * 2^k$
\begin{align*}
T(n) = T(3 * 2^{k-1}) &= 2T(3 2^k)+2\\
&= 4T (3 * 2^{k-2}) + 2 + 4\\
& \vdots\\
&= 2^nT(3)+2+4+...+2^k &| \text{ geometrische Reihe...}\\
&= 3 * 2k + \frac{2^{k+1}-1}{2-1} -1\\
&= 5.2^k -2 = \frac{5}{3}(n)-2\\
\\
T(3) &= T(1) + T(2) + 2\\
&= 0 + 1+ 2\\
&= 3\\
\end{align*} 
\item[a) (alternativ)] Wertebereich verschieben
\begin{align*}
T(n) &= T(\lfloor \frac{n}{2} \rfloor) + T(\lceil \frac{n}{2} \rceil) +2 & | +2\\
\underbrace{T(n) + 2}_{s(n)} &= \underbrace{T(\lfloor \frac{n}{2} \rfloor) + 2}_{s(\lfloor \frac{n}{2} \rfloor)} + \underbrace{T(\lceil \frac{n}{2} \rceil) +2}_{s(\lceil \frac{n}{2} \rceil)}\\
\\
s(n) &= s(3 * 2^k) = s(3)2^k = 5*2^k\\
T(n) &= s(n)-2 = 5.2^k -2 = \frac{5}{3}*n-2\\
\\
s(3) &= T(3) + 2 = 5\\
\end{align*}
\item[b)] z.z.: $1\leq T(n)-T(n-1)\leq 2$
\begin{align*}
\text{Beweis per Ind.:}\quad &\\
\text{Basis:}\quad &T(2)-T(1) = 1\\
(<n) \rightarrow n \quad &T(n)-T(n-1)\\
\text{Fall 1:}\quad\quad & n = 2l	&\text{(Gerade)}\\
\Rightarrow T(n) - T(n-1) &= T(2l) - T(2l-1)\\
&= T(l) + T(l) + 2 = [T(l) + T(l-1) + 2]\\
&= [T(l)-T(l-1)] \underbrace{ \in}_{\text{nach I.V.}} \{1,2\}\\
\\
\text{Fall 2:}\quad\quad & n = 2l + 1	&\text{(analog)}\\
\end{align*}
\item[c)] Idee: siehe Diagramm im Skript. Auf der x-Achse sind $2^k$ Schritte zwischen $2*2^k$ und $3*2^k$ und auf der y-Achse $2*2^k$ Schritte zwischen $3*2^k$ und $3*2^k$\\
\item[d)] z.z. $T(n) = \frac{5}{3}$\\
\textbf{Möglichkeit 1} mit Ind.: z.z.: $T(n) \leq \frac{5}{3} - 2 \quad n \geq 3$\\
\textbf{Wichtig: }Eine stärkere Aussage kann manchmal eher gelingen per Induktion zu gelingen, als eine schwächere. Weil auch eine stärkere Induktionsvoraussetzung benutzt werden kann beim Induktionsschritt.\\
Versuch $T(n) \leq \frac{5}{3} * n$ direkt zu zeigen.\\
\begin{align*}
T(n) &= T(\lfloor\frac{n}{2}\rfloor) + T(\lceil \frac{n}{2} \rceil) +2\\
&\leq \frac{5}{3} (\lfloor\frac{n}{2}\rfloor) + \frac{5}{3}(\lceil \frac{n}{2} \rceil) +2\\
&= \frac{5}{3} (\lfloor\frac{n}{2}\rfloor + \lceil \frac{n}{2} \rceil) + 2\\
&= \frac{5}{3} n+2 &\text{GEHT NICHT! :(}
\end{align*}
Mit $T(n) \leq \frac{5}{3} * n -2$ geht es!
\begin{align*}
T(n) &= T(\lfloor\frac{n}{2}\rfloor) + T(\lceil \frac{n}{2} \rceil) +2\\
&\leq \frac{5}{3} (\lfloor\frac{n}{2}\rfloor)-2 + \frac{5}{3}(\lceil \frac{n}{2} \rceil)-2 +2\\
&= \frac{5}{3} (\lfloor\frac{n}{2}\rfloor + \lceil \frac{n}{2} \rceil) - 2\\
&= \frac{5}{3} n-2 &\text{:)}
\end{align*}
\textbf{Möglichkeit 2:} Argumentativ am Graphen erklären.\\
\textbf{Möglichkeit 3:} direkt $2*2^k \leq n \leq 3*2^k$
\begin{align*}
T(n) &= T(2*2^k) + 2(n-2*2^k)\\
&= 3*2^k-2+2n-4*2^k = 2n-2^k-2\\
&\leq 2*n -\frac{1}{3}n -2 = \frac{5}{3}*n-2\\
\end{align*}
\end{enumerate}

\section*{19. Teilmaxima}
$\underline{5},2,\underline{19},7,\underline{26},\underline{77},56,32$
\begin{enumerate}
\item[a)] mindestens eins (erste Zahl) und maximal $n$, wenn die Liste aufsteigend sortiert ist.
\item[b)] Es gibt insgesamt 8 Elemente. Davon sind zwei Elemente echt kleiner als 7. Die zwei Elemente können wir streichen. o.b.d.A reicht es 6 Elemente zu betrachten und anzunehmen, dass 7 das kleinste Element ist. Daraus folgt, dass die Wahrscheinlichkeit $\frac{1}{6}$ ist. Die 7 muss immer an erster Stelle stehen!\\
\item[c)] Unter den ersten 4 Elementen ist eines das größte.\\
An Pos 4 steht TM $\Leftrightarrow$ Es ist das Größte unter $a_1,...,a_n$\\
Jede Position hat die gleiche Wahrscheinlichkeit, das größte Element unter den ersten 4 zu haben.\\
$\Rightarrow \frac{1}{4}$\\
\item[d)] Wahrscheinlichkeit, dass $i$ ein Teilmaxima ist ist $\frac{1}{i}$ und dass das $k$ größte ist $=\frac{1}{k}$\\
\item[e)] 
\end{enumerate}

\section*{20. Verschiebung des Parameter- und Wertebereichs}
\begin{enumerate}
\item[a)] $A = -5$
\begin{align*}
f(n)&=2f(\lfloor \frac{n+3}{2}\rfloor)-5\\
g(n)&=f(n)+A\\
&=f(n)-5\\
&=2f(\lfloor \frac{n+3}{2}\rfloor)-5-5\\
&=2(f(\lfloor \frac{n+3}{2}\rfloor)-5)\\
&=2(g(\lfloor \frac{n+3}{2}\rfloor))\\
\end{align*}
\item[b)] $B = 3$
\begin{align*}
g(n)&=2g(\lfloor \frac{n+3}{2}\rfloor)\\
g(n+3)&=2g(\lfloor \frac{n+6}{2}\rfloor)\\
&=2g(\lfloor \frac{n}{2}\rfloor+3)\\
\Rightarrow h(n) &= g(n+3)\\
&= 2h(\lfloor \frac{n}{2}\rfloor)
\end{align*}
(Test-)Werte berechnen...
\begin{align*}
\Rightarrow h(0) &= g(0+3) = f(1) - 5 = 1\\
h(1) &= 2h(0) = 2\\
h(2) &= 2h(1) = 4\\
\vdots\\
h(4) &= 2h(2) = 8\\
\vdots\\
h(8) &= 2h(4) =16\\
\end{align*}
Wir raten für $h(n) = 2^{\lfloor \log_2 n+1 \rfloor}$:\\
\begin{align*}
h(n) &= 2h({\lfloor \frac{n}{2} \rfloor})\\
2^{\lfloor \log_2 n +1 \rfloor} &= 2*2^{\lfloor \log_2 \lfloor{\frac{n}{2}}\rfloor +1\rfloor}\\
2^{\lfloor \log_2 n \rfloor} &= 2^{\lfloor \log_2\lfloor \frac{n}{2}\rfloor+1\rfloor}&| \log_b \frac{x}{b} = \log_b x -1\\
&= 2 * 2^{\lfloor \log_2 n-1\rfloor} \\
&= 2^{\lfloor \log_2 n\rfloor}\\
\Rightarrow h(n) &= 2^{\lfloor \log_2 n\rfloor}\\
\end{align*}
\textbf{Anmerkung: }Wegen der ganzzahligen Basis können wir $\lfloor \log_2 \lfloor n-1 \rfloor +1\rfloor$ in $\lfloor \log_2  n-1  +1\rfloor$ umformen, weil die Rundung innerhalb der Rundung keinen Einfluss hat.\\
Formeln für
\begin{align*}
g(n) &= h(n)-3 = 2^{\lfloor \log_2 n\rfloor}-3\\
f(n) &= g(n)+5 = h(n) + 2 = 2^{\lfloor \log_2 n\rfloor} + 2\\
\end{align*}
\item[c)] gegeben:
\begin{align*}
q(n) = q (\lfloor \frac{n+3}{2} \rfloor) +1 (\text{für } n > 4), q(1) = q(2) = q (3) = 1
\end{align*}
Verschieben vom Wertebereich:
\begin{align*}
q(n) &= q (\lfloor \frac{n+3}{2} \rfloor) +1\\
r(n) &= q(n)+A\\
\text{wähle } A &= -1\\
&= r(\lfloor \frac{n+3}{2} \rfloor)\\
s(n) &= r(n-B)\\
\text{wähle } B &= -3\\
s(n) &= r(n-3)\\
&= s(\lfloor \frac{n}{2} \rfloor)
\end{align*}
$s(n)$ lässt sich als folgende Formel umschreiben: $s(n) = \lfloor \frac{n}{2^n} \rfloor = \lfloor 2^{-n} * n \rfloor$
\begin{align*}
r(n) &= \lfloor (2^{-n+3} * n+3) \rfloor\\
q(n) &= \lfloor (2^{-n+3} * n+3) \rfloor+1
\end{align*}
\end{enumerate}







\newpage





.
\end{document}

