\documentclass[ngerman,a4paper]{report}
\usepackage[ngerman]{babel}
\usepackage[T1]{fontenc}
\usepackage[utf8]{inputenc}
%\usepackage{MyriadPro}
\usepackage[scaled]{beramono}
\newcommand\Small{\fontsize{10.5}{10.5}\selectfont}
\newcommand*\LSTfont{\Small\ttfamily\SetTracking{encoding=*}{-20}\lsstyle}
\usepackage{microtype}
\usepackage{geometry}
\geometry{verbose,tmargin=3cm,bmargin=3cm,lmargin=3cm,rmargin=3cm}
\usepackage{listings}
\usepackage{stmaryrd}
\usepackage{paralist}
\usepackage{array}
\usepackage{color}
\usepackage{graphicx}
\usepackage{caption}
\usepackage{url}
\usepackage{amsmath}
\usepackage{accents}
\usepackage{tikz}
\usetikzlibrary{plotmarks}

% Code Listing Style
\definecolor{darkblue}{rgb}{0,0,.6}
\definecolor{darkgreen}{rgb}{0,0.5,0}
\definecolor{darkred}{rgb}{0.5,0,0}
\lstset{%
	language=[x86masm]Assembler, 
	basicstyle=\LSTfont,
	commentstyle=\color{darkgreen}, 
	keywordstyle=\color{darkblue}\bfseries, 
	breaklines=true,
	tabsize=2,
	xleftmargin=\fboxsep,
	xrightmargin=-\fboxsep,
	numbers=left,
	numberstyle=\tiny\color{gray},
	stepnumber=1,
	numbersep=5pt,
	frame=bt,
	stringstyle=\color{darkred},
	showstringspaces=false,
	rulecolor= \color{gray},
	emph=[1]%
	{%  
	    then, not, for, return%
	},
	emphstyle=[1]{\color{darkblue}\bfseries},
	emph=[2]%
	{%  Datatypes
	    %
	},
	emphstyle=[2]{\color{darkblue}\bfseries},
	emph=[3]%
	{%  
	    %
	},
	emphstyle=[3]{\color{darkred}\bfseries},
	literate=%
	{Ö}{{\"O}}1
	{Ä}{{\"A}}1
	{Ü}{{\"U}}1
	{ß}{{\ss}}2
	{ü}{{\"u}}1
	{ä}{{\"a}}1
	{ö}{{\"o}}1
}
\providecommand{\tabularnewline}{\\}

\usepackage{fancyhdr}
\pagestyle{fancy}
\usepackage{lastpage}
\makeatletter

\lhead{\textbf{\@title} \\ \@author}
\chead{}
\rhead{\@date \\ \thepage \ von \pageref{LastPage} }
\cfoot{}

\renewcommand{\maketitle}{}
\newcommand{\utilde}[1]{\underaccent{\tilde}{#1}}
\renewcommand{\familydefault}{\sfdefault}
 
\author{Tobias Höppner, Marian Sigler}
\title{HA - Übung 03. }
\date{Wintersemester 2014/2015}

\begin{document} 
\maketitle 
\section*{18. Rekursion}
Siehe handschriftliche Aufzeichnungen (Seite 3) von Marian. Danke!
\section*{20. Verschiebung des Parameter- und Wertebereichs}
\begin{enumerate}
\item[a)] $A = -5$
\begin{align*}
f(n)&=2f(\lfloor \frac{n+3}{2}\rfloor)-5\\
g(n)&=f(n)+A\\
&=f(n)-5\\
&=2f(\lfloor \frac{n+3}{2}\rfloor)-5-5\\
&=2(f(\lfloor \frac{n+3}{2}\rfloor)-5)\\
&=2(g(\lfloor \frac{n+3}{2}\rfloor))\\
\end{align*}
\item[b)] $B = 3$
\begin{align*}
g(n)&=2g(\lfloor \frac{n+3}{2}\rfloor)\\
g(n+3)&=2g(\lfloor \frac{n+6}{2}\rfloor)\\
&=2g(\lfloor \frac{n}{2}\rfloor+3)\\
\Rightarrow h(n) &= g(n+3)\\
&= 2h(\lfloor \frac{n}{2}\rfloor)
\end{align*}
(Test-)Werte berechnen...
\begin{align*}
\Rightarrow h(0) &= g(0+3) = f(1) - 5 = 1\\
h(1) &= 2h(0) = 2\\
h(2) &= 2h(1) = 4\\
\vdots\\
h(4) &= 2h(2) = 8\\
\vdots\\
h(8) &= 2h(4) =16\\
\end{align*}
Wir raten für $h(n) = 2^{\lfloor \log_2 n+1 \rfloor}$:\\
\begin{align*}
h(n) &= 2h({\lfloor \frac{n}{2} \rfloor})\\
2^{\lfloor \log_2 n +1 \rfloor} &= 2*2^{\lfloor \log_2 \lfloor{\frac{n}{2}}\rfloor +1\rfloor}\\
2^{\lfloor \log_2 n \rfloor} &= 2^{\lfloor \log_2\lfloor \frac{n}{2}\rfloor+1\rfloor}&| \log_b \frac{x}{b} = \log_b x -1\\
&= 2 * 2^{\lfloor \log_2 n-1\rfloor} \\
&= 2^{\lfloor \log_2 n\rfloor}\\
\Rightarrow h(n) &= 2^{\lfloor \log_2 n\rfloor}\\
\end{align*}
\textbf{Anmerkung: }Wegen der ganzzahligen Basis können wir $\lfloor \log_2 \lfloor n-1 \rfloor +1\rfloor$ in $\lfloor \log_2  n-1  +1\rfloor$ umformen, weil die Rundung innerhalb der Rundung keinen Einfluss hat.\\
Formeln für
\begin{align*}
g(n) &= h(n)-3 = 2^{\lfloor \log_2 n\rfloor}-3\\
f(n) &= g(n)+5 = h(n) + 2 = 2^{\lfloor \log_2 n\rfloor} + 2\\
\end{align*}
\item[c)] gegeben:
\begin{align*}
q(n) = q (\lfloor \frac{n+3}{2} \rfloor) +1 (\text{für } n > 4), q(1) = q(2) = q (3) = 1
\end{align*}
Verschieben vom Wertebereich:
\begin{align*}
q(n) &= q (\lfloor \frac{n+3}{2} \rfloor) +1\\
r(n) &= q(n)+A\\
\text{wähle } A &= -1\\
&= r(\lfloor \frac{n+3}{2} \rfloor)\\
s(n) &= r(n-B)\\
\text{wähle } B &= -3\\
s(n) &= r(n-3)\\
&= s(\lfloor \frac{n}{2} \rfloor)
\end{align*}
$s(n)$ lässt sich als folgende Formel umschreiben: $s(n) = \lfloor \frac{n}{2^n} \rfloor = \lfloor 2^{-n} * n \rfloor$
\begin{align*}
r(n) &= \lfloor (2^{-n+3} * n+3) \rfloor\\
q(n) &= \lfloor (2^{-n+3} * n+3) \rfloor+1
\end{align*}
\end{enumerate}







\newpage





.
\end{document}

