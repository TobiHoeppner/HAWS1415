\documentclass[ngerman,a4paper]{report}
\usepackage[ngerman]{babel}
\usepackage[T1]{fontenc}
\usepackage[utf8]{inputenc}
%\usepackage{MyriadPro}
\usepackage[scaled]{beramono}
\newcommand\Small{\fontsize{10.5}{10.5}\selectfont}
\newcommand*\LSTfont{\Small\ttfamily\SetTracking{encoding=*}{-20}\lsstyle}
\usepackage{microtype}
\usepackage{geometry}
\geometry{verbose,tmargin=3cm,bmargin=3cm,lmargin=3cm,rmargin=3cm}
\usepackage{listings}
\usepackage{stmaryrd}
\usepackage{paralist}
\usepackage{array}
\usepackage{color}
\usepackage{graphicx}
\usepackage{caption}
\usepackage{url}
\usepackage{amsmath}
\usepackage{accents}
\usepackage{tikz}
\usetikzlibrary{plotmarks}

% Code Listing Style
\definecolor{darkblue}{rgb}{0,0,.6}
\definecolor{darkgreen}{rgb}{0,0.5,0}
\definecolor{darkred}{rgb}{0.5,0,0}
\lstset{%
	language=Python, 
	basicstyle=\LSTfont,
	commentstyle=\color{darkgreen}, 
	keywordstyle=\color{darkblue}\bfseries, 
	breaklines=true,
	tabsize=2,
	xleftmargin=\fboxsep,
	xrightmargin=-\fboxsep,
	numbers=left,
	numberstyle=\tiny\color{gray},
	stepnumber=1,
	numbersep=5pt,
	frame=bt,
	stringstyle=\color{darkred},
	showstringspaces=false,
	rulecolor= \color{gray},
	emph=[1]%
	{%  
	    then, not, for, return%
	},
	emphstyle=[1]{\color{darkblue}\bfseries},
	emph=[2]%
	{%  Datatypes
	    %
	},
	emphstyle=[2]{\color{darkblue}\bfseries},
	emph=[3]%
	{%  
	    %
	},
	emphstyle=[3]{\color{darkred}\bfseries},
	literate=%
	{Ö}{{\"O}}1
	{Ä}{{\"A}}1
	{Ü}{{\"U}}1
	{ß}{{\ss}}2
	{ü}{{\"u}}1
	{ä}{{\"a}}1
	{ö}{{\"o}}1
}
\providecommand{\tabularnewline}{\\}

\usepackage{fancyhdr}
\pagestyle{fancy}
\usepackage{lastpage}
\makeatletter

\lhead{\textbf{\@title} \\ \@author}
\chead{}
\rhead{\@date \\ \thepage \ von \pageref{LastPage} }
\cfoot{}

\renewcommand{\maketitle}{}
\newcommand{\utilde}[1]{\underaccent{\tilde}{#1}}
\renewcommand{\familydefault}{\sfdefault}
 
\author{Tobias Höppner}
\title{HA - Gummi-Übung 06.}
\date{Wintersemester 2014/2015}

\begin{document} 
\maketitle 

\subsubsection*{35. Wechselgeld, 10 Punkte}
\textbf{Pseudocode:} Naive Idee, einfach formuliert:
Eingabe: Betrag, Geldwerte()
\begin{enumerate}
\item sortiere Geldwerte absteigend, mach aus Geldwerte gleichzeitig Tupel \lstinline!(Geldwert, Anzahl)!
\item \lstinline!Restbetrag = Betrag!
\item für jeden \lstinline!Geldwert!
\begin{enumerate}
\item wenn \lstinline!Betrag - Geldwert < 0!, dann nehme nächsten \lstinline!Geldwert!
\item ziehe solange \lstinline!Geldwert! vom Restbetrag ab bis \lstinline!Restbetrag < Geldwert!
\end{enumerate}
\item ist der \lstinline!Restbetrag == 0!, dann gebe verwendete Geldwerte zurück
\end{enumerate}
Beste Lösung: nutze die mathematische Operation \lstinline!divmod(x,y)!. 
\textbf{Code}
\begin{lstlisting}
def money(b, values):
    r = b
    l = []
    for k in sorted(values, reverse=True):
        c, r = divmod(r, k)
        l.append((k, c))
    if r == 0:
        a = str(b)+'='
        for b in l:
            for i in range(0, b[1]):
                a += str(b[0]) + '+'
        print(a[0:len(a)-1])
        return b, l
    print('keine Lösung')
    return b, []
\end{lstlisting}
\textbf{Laufzeit}: $n = $Geldwerte (Länge der Liste)
\begin{enumerate}
\item[-] sortieren: $\mathcal{O}(n * \log n)$
\item[-] Werte berechnen: $\mathcal{O}(n)$
\item[-] Ausgabe: $\mathcal{O}(n * $\lstinline!Anzahl berechneter Geldwerte!$)$
\end{enumerate}
Ohne die Ausgabe also $\mathcal{O}(n * \log n)$. Die Ausgabe, wie sie in der Aufgabenstellung vorhanden ist, kann die Laufzeit drastisch erhöhen.\\
\subsubsection*{38. Beste unabhängige Knotenmenge in Bäumen, 10 Punkte}
Durchlaufe Baum (G = (V,E)) mit Tiefensuche und markiere (ggfs. teile gleich in getrennte Listen auf) die Knoten abwechseln. Die resultierenden Knotenmengen erfüllen folgende Eigenschaft: $\forall v, v' \in V'. v \neq v' \Rightarrow \{v,v' \} \notin E$.
Anschließend berechne die Summe beider Teilmengen und gebe die größte Menge zurück.

\end{document}
