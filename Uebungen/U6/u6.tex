\documentclass[ngerman,a4paper]{report}
\usepackage[ngerman]{babel}
\usepackage[T1]{fontenc}
\usepackage[utf8]{inputenc}
\usepackage{MyriadPro}
\usepackage[scaled]{beramono}
\newcommand\Small{\fontsize{10.5}{10.5}\selectfont}
\newcommand*\LSTfont{\Small\ttfamily\SetTracking{encoding=*}{-20}\lsstyle}
\usepackage{microtype}
\usepackage{geometry}
\geometry{verbose,tmargin=3cm,bmargin=3cm,lmargin=3cm,rmargin=3cm}
\usepackage{listings}
\usepackage{stmaryrd}
\usepackage{paralist}
\usepackage{array}
\usepackage{color}
\usepackage{graphicx}
\usepackage{caption}
\usepackage{url}
\usepackage{amsmath}
\usepackage{accents}
\usepackage{tikz}
\usetikzlibrary{plotmarks}

% Code Listing Style
\definecolor{darkblue}{rgb}{0,0,.6}
\definecolor{darkgreen}{rgb}{0,0.5,0}
\definecolor{darkred}{rgb}{0.5,0,0}
\lstset{%
	language=Python, 
	basicstyle=\LSTfont,
	commentstyle=\color{darkgreen}, 
	keywordstyle=\color{darkblue}\bfseries, 
	breaklines=true,
	tabsize=2,
	xleftmargin=\fboxsep,
	xrightmargin=-\fboxsep,
	numbers=left,
	numberstyle=\tiny\color{gray},
	stepnumber=1,
	numbersep=5pt,
	frame=bt,
	stringstyle=\color{darkred},
	showstringspaces=false,
	rulecolor= \color{gray},
	emph=[1]%
	{%  
	    then, not, for, return%
	},
	emphstyle=[1]{\color{darkblue}\bfseries},
	emph=[2]%
	{%  Datatypes
	    %
	},
	emphstyle=[2]{\color{darkblue}\bfseries},
	emph=[3]%
	{%  
	    %
	},
	emphstyle=[3]{\color{darkred}\bfseries},
	literate=%
	{Ö}{{\"O}}1
	{Ä}{{\"A}}1
	{Ü}{{\"U}}1
	{ß}{{\ss}}2
	{ü}{{\"u}}1
	{ä}{{\"a}}1
	{ö}{{\"o}}1
}
\providecommand{\tabularnewline}{\\}

\usepackage{fancyhdr}
\pagestyle{fancy}
\usepackage{lastpage}
\makeatletter

\lhead{\textbf{\@title} \\ \@author}
\chead{}
\rhead{\@date \\ \thepage \ von \pageref{LastPage} }
\cfoot{}

\renewcommand{\maketitle}{}
\newcommand{\utilde}[1]{\underaccent{\tilde}{#1}}
\renewcommand{\familydefault}{\sfdefault}
 
\author{Tobias Höppner}
\title{HA - Gummi-Übung 06.}
\date{Wintersemester 2014/2015}

\begin{document} 
\maketitle 

\subsection*{35. Wechselgeld, 10 Punkte}
%\textbf{Pseudocode:} Naive Idee, einfach formuliert:
%Eingabe: Betrag, Geldwerte()
%\begin{enumerate}
%\item sortiere Geldwerte absteigend, mach aus Geldwerte gleichzeitig Tupel \lstinline!(Geldwert, Anzahl)!
%\item \lstinline!Restbetrag = Betrag!
%\item für jeden \lstinline!Geldwert!
%\begin{enumerate}
%\item wenn \lstinline!Betrag - Geldwert < 0!, dann nehme nächsten \lstinline!Geldwert!
%\item ziehe solange \lstinline!Geldwert! vom Restbetrag ab bis \lstinline!Restbetrag < Geldwert!
%\end{enumerate}
%\item ist der \lstinline!Restbetrag == 0!, dann gebe verwendete Geldwerte zurück
%\end{enumerate}
%Beste Lösung: nutze die mathematische Operation \lstinline!divmod(x,y)!. 
%\textbf{Code}
%\begin{lstlisting}
%def money(b, values):
%    r = b
%    l = []
%    for k in sorted(values, reverse=True):
%        c, r = divmod(r, k)
%        l.append((k, c))
%    if r == 0:
%        a = str(b)+'='
%        for b in l:
%            for i in range(0, b[1]):
%                a += str(b[0]) + '+'
%        print(a[0:len(a)-1])
%        return b, l
%    print('keine Lösung')
%    return b, []
%\end{lstlisting}
%\textbf{Laufzeit}: $n = $Geldwerte (Länge der Liste)
%\begin{enumerate}
%\item[-] sortieren: $\mathcal{O}(n * \log n)$
%\item[-] Werte berechnen: $\mathcal{O}(n)$ unter Annahme, dass \lstinline!divmod! in $\mathcal{O}(1)$ ausgeführt wird.
%\item[-] Ausgabe: $\mathcal{O}(n * $\lstinline!Anzahl berechneter Geldwerte!$)$
%\end{enumerate}
%Ohne die Ausgabe also $\mathcal{O}(n * \log n)$. Die Ausgabe, wie sie in der Aufgabenstellung vorhanden ist, kann die Laufzeit drastisch erhöhen.\\
Idee: liste sortieren, mit divmod(Betrag, Geldwert) Anzahl der Münzen und Restbetrag bestimmen, den Geldwert Anzahl-mal auf einen Stack speichern. Sollte die Anzahl 0 sein, dann divmod(Betrag, nächster Geldwert), wenn diese Anzahl ebenfalls 0 ist, dann hole eine Münze vom Stack, addiere Wert auf Betrag und berechne divmod(Betrag, Geldwert).\\
Eingabe: Betrag, Geldwerte(Liste, unsortiert)
\begin{enumerate}
\item sortiere Geldwerte absteigend
\item \lstinline!Restbetrag = Betrag!
\item für jeden Geldwert \lstinline!g! aus Liste
\begin{enumerate}
\item berechne \lstinline!Anzahl, Restbetrag = divmod(Restbetrag, g)!
\item wenn \lstinline!Anzahl == 0! und der Stack ist nicht leer und \lstinline!a==0! aus \lstinline!a,r = divmod(Restbetrag, g.next())!, dann \lstinline!Restbetrag + !letzter Stackwert
\item berechne \lstinline!Anzahl, Restbetrag = divmod(Restbetrag, g)!
\item lege Geldwert \lstinline!g Anzahl!-mal auf Stack
\end{enumerate}
\item Wenn \lstinline!Restbetrag = 0!, dann gebe alle Elemente vom Stack zurück, sonst gebe \lstinline!"Nicht lösbar."! zurück
\end{enumerate}
\subsubsection*{Laufzeit}
$\mathcal{O}(c*n)$ wobei $n$ die Anzahl der Geldwerte ist. Teuerste Operation ist das Speichern auf dem Stack.
\subsubsection{Speicherbedarf}
$\mathcal{O}(c*n)$ wobei $n$ die Anzahl der Geldwerte ist und $c$ die Anzahl der verwendeten Münzen.
\subsection*{38. Beste unabhängige Knotenmenge in Bäumen, 10 Punkte}
\subsubsection*{Teilprobleme}
Ein Knoten ist Teil der besten unabhängigen Knotenmenge (folgend BUK), wenn sein Wert die Gesamtmenge erhöht. Ist ein Knoten Teil der BUK, dann können nur noch die Enkelkinder zur BUK gehören. Wenn er kein Teil der BUK wird, dann müssen seine Kinder Teil der BUK sein. Ziel ist es für jeden Knoten (von der Wurzel aus startend) die maximale BUK zu finden.
\subsubsection*{Rekursion}
\begin{align*}
\text{\lstinline!BUK!}(K) = \max \{ \sum \text{\lstinline!BUK!(Kinder von $K$)}, \sum \text{\lstinline!BUK!(Enkelkinder von $K$)} \}
\end{align*}

Die Funktion muss abbrechen sobald man an den Blättern (Wert vom Blatt zurückgeben) angekommen ist.

\subsubsection*{Laufzeit}
$\mathcal{O}(2^n)$, wobei $n$ die Anzahl der Knoten ist. Man rechnet mit dieser Rekursion mehrfach alle Teilergebnisse aus. Besser wäre es, diese Teilergebnisse in eine Tabelle zu Speichern und die Rekursion dahingehend zu erweitern, dass bereits berechnete Teilergebnisse berücksichtigt werden.

\subsubsection*{Speicher}
Diese naive Lösung verwendet keine Speicherstruktur. Daher $\mathcal{O}(n)$.

\end{document}
