\documentclass[ngerman,a4paper]{report}
\usepackage[ngerman]{babel}
\usepackage[T1]{fontenc}
\usepackage[utf8]{inputenc}
%\usepackage{MyriadPro}
\usepackage[scaled]{beramono}
\newcommand\Small{\fontsize{10.5}{10.5}\selectfont}
\newcommand*\LSTfont{\Small\ttfamily\SetTracking{encoding=*}{-20}\lsstyle}
\usepackage{microtype}
\usepackage{geometry}
\geometry{verbose,tmargin=3cm,bmargin=3cm,lmargin=3cm,rmargin=3cm}
\usepackage{listings}
\usepackage{stmaryrd}
\usepackage{paralist}
\usepackage{array}
\usepackage{color}
\usepackage{graphicx}
\usepackage{caption}
\usepackage{url}
\usepackage{amsmath}
\usepackage{amssymb}
\usepackage{accents}
\usepackage{tikz}
\usetikzlibrary{arrows}

%tikz
\tikzset{
  treenode/.style = {align=center, inner sep=0pt, text centered,
    font=\sffamily},
  arn_n/.style = {treenode, circle, white, font=\sffamily\bfseries, draw=black,
    fill=black, text width=3.5em},% arbre rouge noir, noeud noir
  arn_r/.style = {treenode, red, 
    text width=1.5em, very thick},% arbre rouge noir, noeud rouge
  arn_x/.style = {treenode, rectangle, draw=black,
    minimum width=0.5em, minimum height=0.5em}% arbre rouge noir, nil
}

% Code Listing Style
\definecolor{darkblue}{rgb}{0,0,.6}
\definecolor{darkgreen}{rgb}{0,0.5,0}
\definecolor{darkred}{rgb}{0.5,0,0}
\lstset{%
	language=Python, 
	basicstyle=\LSTfont,
	commentstyle=\color{darkgreen}, 
	keywordstyle=\color{darkblue}\bfseries, 
	breaklines=true,
	tabsize=2,
	xleftmargin=\fboxsep,
	xrightmargin=-\fboxsep,
	numbers=left,
	numberstyle=\tiny\color{gray},
	stepnumber=1,
	numbersep=5pt,
	frame=bt,
	stringstyle=\color{darkred},
	showstringspaces=false,
	rulecolor= \color{gray},
	emph=[1]%
	{%  
	    then, not, for, return%
	},
	emphstyle=[1]{\color{darkblue}\bfseries},
	emph=[2]%
	{%  Datatypes
	    %
	},
	emphstyle=[2]{\color{darkblue}\bfseries},
	emph=[3]%
	{%  
	    %
	},
	emphstyle=[3]{\color{darkred}\bfseries},
	literate=%
	{Ö}{{\"O}}1
	{Ä}{{\"A}}1
	{Ü}{{\"U}}1
	{ß}{{\ss}}2
	{ü}{{\"u}}1
	{ä}{{\"a}}1
	{ö}{{\"o}}1
}
\providecommand{\tabularnewline}{\\}

\usepackage{fancyhdr}
\pagestyle{fancy}
\usepackage{lastpage}
\makeatletter

\lhead{\textbf{\@title} \\ \@author}
\chead{}
\rhead{\@date \\ \thepage \ von \pageref{LastPage} }
\cfoot{}

\renewcommand{\maketitle}{}
\newcommand{\utilde}[1]{\underaccent{\tilde}{#1}}
\renewcommand{\familydefault}{\sfdefault}
 
\author{Tobias Höppner}
\title{HA - Beton-Übung 13.}
\date{Wintersemester 2014/2015}

\begin{document} 
\maketitle 

\subsubsection*{72. Kürzester einfacher Weg, 10 Punkte}
Funktion $f$: 
\begin{enumerate}
\item wähle $s,t \in V$, sodass $s$ mit $t$ mit einer gerichteten Kante von $t$ nach $s$ verbunden ist.
\item wähle $k = - |V|$
\item für alle Kantengewichte setze $c_{ij} = -1$
\end{enumerate}
Berechne KEW mit diesen Parameter. \\

\begin{enumerate}
\item KEW $\in$ NP? Ja, das Zertifikat, Weg $s,t$ mit  $\sum c_{ij} \leq k$, ist in polynomieller Zeit überprüfbar.
\item Eingabe $x \in$ HAM $\rightarrow f(x) \in $ KEW \\
Dadurch, dass ich der Eingabe nur Kantengewichte hinzufüge, die den Graphen aber nicht verändern. Wird das Problem nicht verändert.
\item Eingabe $x \in$ HAM $\leftarrow f(x) \in $ KEW\\
Wenn $f(x)$ eine Lösung hat, dann erhalte ich ein neues Zertifikat für HAM.
\end{enumerate}

%\subsubsection*{73. Klausuraufgaben, 5 Betonpunkte}
%Wofür steht NP?
%\begin{enumerate}
%\item[$\square$] Null Problemo!
%\item[$\square$] Niedrig Preise!
%\item[$\square$] not persistent.
%\item[$\square$] null pointer.
%\item[$\square$] Nicht prima.
%\item[$\square$] Nicht primzahlzerlegbar.
%\item[$\square$] Nicht-deterministisch berechenbar.
%\item[$\square$] Nicht-deterministisch praxisnahberechenbar.
%\item[$\square$] Nicht-deterministisch polynomiell berechenbar.
%\item[$\square$] Nur bei Pfollmond berechenbar.
%\item[$\square$] Nur partiell berechenbar.
%\item[$\square$] Nichtmal deine Mutti kann das ausrechnen.
%\item[$\square$] Niemals! Peter!
%\item[$\square$] Nur Prollos wissen, wie das geht.
%\item[$\square$] Neue Probleme.
%\item[$\square$] Natriumpolyamid.
%\item[$\square$] Nicht Pfändbar.
%\item[$\square$] No Pattern.
%\item[$\square$] No Patternmatching.
%\item[$\square$] Nicht (im stehen) Pinkeln.
%\item[$\square$] Nitratpegel.
%\end{enumerate}

\subsubsection*{74. Das Mengenüberdeckungsproblem, Selbstreduktion, 10 Punkte}
Dadurch, dass $F$ zur Eingabe von $X$ gehört. Muss $F$ polynomiell Groß sein. Demnach ist es möglich $U$ in polynomieller Zeit zu finden. Die Überdeckung $U$ finde ich wie folgt:
\begin{enumerate}
\item sortiere Teilmengen der Größe nach, absteigend.
\item für alle Teilmengen $A_i$, prüfe mit $X$, ob es eine Überdeckung mit den anderen Teilmengen gibt
\item wenn es eine Überdeckung gibt, dann entferne Teilmenge aus der Auswahl ($A_i$ beinhaltet diese Menge bereits).
\item wenn nicht, nehme nächste Teilmenge und prüfe wieder.
\item breche ab, wenn alle Teilmengen überpüft wurden.
\item Prüfe mit $X$, ob alle Teilmengen zusammen ($U$) eine Überdeckung von $S$ liefern.
\end{enumerate}

\end{document}
