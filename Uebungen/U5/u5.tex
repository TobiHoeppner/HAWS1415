\documentclass[ngerman,a4paper]{report}
\usepackage[ngerman]{babel}
\usepackage[T1]{fontenc}
\usepackage[utf8]{inputenc}
%\usepackage{MyriadPro}
\usepackage[scaled]{beramono}
\newcommand\Small{\fontsize{10.5}{10.5}\selectfont}
\newcommand*\LSTfont{\Small\ttfamily\SetTracking{encoding=*}{-20}\lsstyle}
\usepackage{microtype}
\usepackage{geometry}
\geometry{verbose,tmargin=3cm,bmargin=3cm,lmargin=3cm,rmargin=3cm}
\usepackage{listings}
\usepackage{stmaryrd}
\usepackage{paralist}
\usepackage{array}
\usepackage{color}
\usepackage{graphicx}
\usepackage{caption}
\usepackage{url}
\usepackage{amsmath}
\usepackage{accents}
\usepackage{tikz}
\usetikzlibrary{plotmarks}

% Code Listing Style
\definecolor{darkblue}{rgb}{0,0,.6}
\definecolor{darkgreen}{rgb}{0,0.5,0}
\definecolor{darkred}{rgb}{0.5,0,0}
\lstset{%
	language=[x86masm]Assembler, 
	basicstyle=\LSTfont,
	commentstyle=\color{darkgreen}, 
	keywordstyle=\color{darkblue}\bfseries, 
	breaklines=true,
	tabsize=2,
	xleftmargin=\fboxsep,
	xrightmargin=-\fboxsep,
	numbers=left,
	numberstyle=\tiny\color{gray},
	stepnumber=1,
	numbersep=5pt,
	frame=bt,
	stringstyle=\color{darkred},
	showstringspaces=false,
	rulecolor= \color{gray},
	emph=[1]%
	{%  
	    then, not, for, return%
	},
	emphstyle=[1]{\color{darkblue}\bfseries},
	emph=[2]%
	{%  Datatypes
	    %
	},
	emphstyle=[2]{\color{darkblue}\bfseries},
	emph=[3]%
	{%  
	    %
	},
	emphstyle=[3]{\color{darkred}\bfseries},
	literate=%
	{Ö}{{\"O}}1
	{Ä}{{\"A}}1
	{Ü}{{\"U}}1
	{ß}{{\ss}}2
	{ü}{{\"u}}1
	{ä}{{\"a}}1
	{ö}{{\"o}}1
}
\providecommand{\tabularnewline}{\\}

\usepackage{fancyhdr}
\pagestyle{fancy}
\usepackage{lastpage}
\makeatletter

\lhead{\textbf{\@title} \\ \@author}
\chead{}
\rhead{\@date \\ \thepage \ von \pageref{LastPage} }
\cfoot{}

\renewcommand{\maketitle}{}
\newcommand{\utilde}[1]{\underaccent{\tilde}{#1}}
\renewcommand{\familydefault}{\sfdefault}
 
\author{Tobias Höppner}
\title{HA - Übung 05.}
\date{Wintersemester 2014/2015}

\begin{document} 
\maketitle 

\subsection*{29. Programmieraufgabe: Multiplikation langer Zahlen nach Karatsuba, 20 Punkte}
Daten:
\begin{tabular}{|c|c|}
\hline
N & ms\\
\hline
1000 & 7\\
\hline
2000 & 13\\
\hline
3000 & 22\\
\hline
4000 & 41\\
\hline
5000 & 43\\
\hline
6000 & 67\\
\hline
7000 & 118\\
\hline
8000 & 122\\
\hline
9000 & 123\\
\hline
10000 & 133\\
\hline
11000 & 143\\
\hline
12000 & 202\\
\hline
13000 & 303\\
\hline
14000 & 359\\
\hline
15000 & 371\\
\hline
16000 & 373\\
\hline
17000 & 377\\
\hline
18000 & 379\\
\hline
19000 & 384\\
\hline
20000 & 405\\
\hline
21000 & 402\\
\hline
22000 & 434\\
\hline
23000 & 512\\
\hline
24000 & 616\\
\hline
25000 & 764\\
\hline
26000 & 925\\
\hline
27000 & 1036\\
\hline
28000 & 1090\\
\hline
29000 & 1113\\
\hline
30000 & 1121\\
\hline
31000 & 1115\\
\hline
32000 & 1123\\
\hline
33000 & 1120\\
\hline
34000 & 1117\\
\hline
35000 & 1137\\
\hline
36000 & 1121\\
\hline
37000 & 1132\\
\hline
38000 & 1145\\
\hline
39000 & 1156\\
\hline
40000 & 1212\\
\hline
41000 & 1191\\
\hline
42000 & 1221\\
\hline
43000 & 1285\\
\hline
44000 & 1349\\
\hline
45000 & 1395\\
\hline
46000 & 1533\\
\hline
47000 & 1669\\
\hline
48000 & 1876\\
\hline

\end{tabular}
Diagramm:

\end{document}
