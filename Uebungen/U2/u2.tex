\documentclass[ngerman,a4paper]{report}
\usepackage[ngerman]{babel}
\usepackage[T1]{fontenc}
\usepackage[utf8]{inputenc}
\usepackage{MyriadPro}
\usepackage[scaled]{beramono}
\newcommand\Small{\fontsize{10.5}{10.5}\selectfont}
\newcommand*\LSTfont{\Small\ttfamily\SetTracking{encoding=*}{-20}\lsstyle}
\usepackage{microtype}
\usepackage{geometry}
\geometry{verbose,tmargin=3cm,bmargin=3cm,lmargin=3cm,rmargin=3cm}
\usepackage{listings}
\usepackage{stmaryrd}
\usepackage{paralist}
\usepackage{array}
\usepackage{color}
\usepackage{graphicx}
\usepackage{caption}
\usepackage{url}
\usepackage{amsmath}
\usepackage{accents}
\usepackage{tikz}
\usetikzlibrary{plotmarks}

% Code Listing Style
\definecolor{darkblue}{rgb}{0,0,.6}
\definecolor{darkgreen}{rgb}{0,0.5,0}
\definecolor{darkred}{rgb}{0.5,0,0}
\lstset{%
	language=[x86masm]Assembler, 
	basicstyle=\LSTfont,
	commentstyle=\color{darkgreen}, 
	keywordstyle=\color{darkblue}\bfseries, 
	breaklines=true,
	tabsize=2,
	xleftmargin=\fboxsep,
	xrightmargin=-\fboxsep,
	numbers=left,
	numberstyle=\tiny\color{gray},
	stepnumber=1,
	numbersep=5pt,
	frame=bt,
	stringstyle=\color{darkred},
	showstringspaces=false,
	rulecolor= \color{gray},
	emph=[1]%
	{%  
	    then, not, for, return%
	},
	emphstyle=[1]{\color{darkblue}\bfseries},
	emph=[2]%
	{%  Datatypes
	    %
	},
	emphstyle=[2]{\color{darkblue}\bfseries},
	emph=[3]%
	{%  
	    %
	},
	emphstyle=[3]{\color{darkred}\bfseries},
	literate=%
	{Ö}{{\"O}}1
	{Ä}{{\"A}}1
	{Ü}{{\"U}}1
	{ß}{{\ss}}2
	{ü}{{\"u}}1
	{ä}{{\"a}}1
	{ö}{{\"o}}1
}
\providecommand{\tabularnewline}{\\}

\usepackage{fancyhdr}
\pagestyle{fancy}
\usepackage{lastpage}
\makeatletter

\lhead{\textbf{\@title} \\ \@author}
\chead{}
\rhead{\@date \\ \thepage \ von \pageref{LastPage} }
\cfoot{}

\renewcommand{\maketitle}{}
\newcommand{\utilde}[1]{\underaccent{\tilde}{#1}}
\renewcommand{\familydefault}{\sfdefault}
 
\author{Tobias Höppner}
\title{HA - Übung 02. }
\date{Wintersemester 2014/2015}

\begin{document} 
\maketitle 
\section*{10.  Schmutzige Tricks im Einheitskostenmaß,  10 Punkte}
Gegeben:
\begin{align*}
x &= (x_1,...,x_n)\\
y &= (y_1,...,y_n)\\
z &= (z_1,...,z_n)\\
\\
M &=\max \{|x_i|,|y_i|,|z_i| \; |i=\{1,..,n\}\}\\
u &> 4M\\
a &=\sum\limits_{i=1}^n u^{n-1} x_i\\
b &=\sum\limits_{i=1}^n u^{n-1} y_i\\
c &=\sum\limits_{i=1}^n u^{n-1} z_i\\
\end{align*}

zu Zeigen:\\
$(a+b=c) \Leftrightarrow x+y=z$\\

a.) $\Leftarrow$ - Einfach einsetzen...\\
\begin{align*}
a+b &\overset{Def}{=} \sum\limits_{i=0}^n x_i u^{n-i} + \sum\limits_{i=0}^n y_i u^{n-i}\\
&= \sum\limits_{i=0}^n (x_i+y_i) u^{n-i} \\
&\overset{Def}{=} \sum\limits_{i=0}^n z_i u^{n-i} \\
&= c\\
\end{align*}
c.) $\Rightarrow$\\
Wissen: $a + b = c$\\
Idee: $\Rightarrow a+b = c \mod u$, also irgendwie versuchen die $u$-Werte zu lösen.\\
\begin{align*}
a + b &= u (\sum ...) + (x_n + y_n)\\
c &= u (\sum ...) + (z_n)\\
&\Rightarrow x_n+y_n = z_n \mod U
\end{align*}
%Wir wissen ebenfalls $|x_n+y_n|<u$\\
%und $|z_n| < u$\\
%also $\Rightarrow x_n + y_n = z_n$\\

%$a' = (a-x_n)/u$\\
%$b' = (b-y_n)/u$\\
%$c' = (c-z_n)/u$\\

%$\Rightarrow a'+b' = c'$\\
%$\underset{per Induktion}{\Rightarrow} x_{n-1}+y_{n_1} = z_{n-1}$\\
\newpage
\section*{11.  Programmieren einer Registermaschine, 10 Punkte}
\subsection*{Pseudocode}
\begin{lstlisting}
in (n, x[1,...,n])
    max = x[1]
    tmp = x[1]
    for i = 2, i < n, i++
        tmp = x[i]
        if tmp > max
            tmp = max
    return max
\end{lstlisting}

\subsection*{RAM}
\begin{lstlisting}
; r0 = n
; r1-r5 sind frei (siehe letztes Tutorium)
; r6-r(n+5) enthalten die Eingabe
r1 = 1      ; max_i
r2 = r6    	; max
r3 = r6    	; tmp
r4 = r6     ; i
r5 = 0      ; 
loop:
    r4 = r4 + 1     ; ++i
    r5 = r0 + 6     ; abbruch wenn n+6-i 
    r5 = r5 - r4
    GZ r5, halt

    r3 = r3 + (r4)  ; tmp aktualisieren
    r5 = r3 - r2    
    GGZ R5, newmax
    GOTO loop
newmax:
    r2 = r3         ; neues max setzen
    r1 = r4 - 5     ; neues max_i setzen
    GOTO loop
halt:
	HALT
\end{lstlisting}
\subsection*{Laufzeitanalyse}
\textbf{EKM:} In der Summe $\Theta(n)$\\
\textbf{LKM:} $\mathcal{O} (n L)$\\
L = längster vorkommender Eintrag in einem Register.
\subsection*{Speicherbedarf}
\textbf{EKM:} $\Theta(n)$ - wir benutzen $\Theta(n)$ Register\\
\textbf{LKM:} $\mathcal{O}(nL)$ laut Vorlesung\\
$\Rightarrow n$ Register mit Inhalt $\Omega(L)$\\

\end{document}
