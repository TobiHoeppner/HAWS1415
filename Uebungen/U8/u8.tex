\documentclass[ngerman,a4paper]{report}
\usepackage[ngerman]{babel}
\usepackage[T1]{fontenc}
\usepackage[utf8]{inputenc}
%s\usepackage{MyriadPro}
\usepackage[scaled]{beramono}
\newcommand\Small{\fontsize{10.5}{10.5}\selectfont}
\newcommand*\LSTfont{\Small\ttfamily\SetTracking{encoding=*}{-20}\lsstyle}
\usepackage{microtype}
\usepackage{geometry}
\geometry{verbose,tmargin=3cm,bmargin=3cm,lmargin=3cm,rmargin=3cm}
\usepackage{listings}
\usepackage{stmaryrd}
\usepackage{paralist}
\usepackage{array}
\usepackage{color}
\usepackage{graphicx}
\usepackage{caption}
\usepackage{url}
\usepackage{amsmath}
\usepackage{accents}
\usepackage{tikz}
\usetikzlibrary{arrows}

%tikz
\tikzset{
  treenode/.style = {align=center, inner sep=0pt, text centered,
    font=\sffamily},
  arn_n/.style = {treenode, circle, white, font=\sffamily\bfseries, draw=black,
    fill=black, text width=3.5em},% arbre rouge noir, noeud noir
  arn_r/.style = {treenode, red, 
    text width=1.5em, very thick},% arbre rouge noir, noeud rouge
  arn_x/.style = {treenode, rectangle, draw=black,
    minimum width=0.5em, minimum height=0.5em}% arbre rouge noir, nil
}

% Code Listing Style
\definecolor{darkblue}{rgb}{0,0,.6}
\definecolor{darkgreen}{rgb}{0,0.5,0}
\definecolor{darkred}{rgb}{0.5,0,0}
\lstset{%
	language=Python, 
	basicstyle=\LSTfont,
	commentstyle=\color{darkgreen}, 
	keywordstyle=\color{darkblue}\bfseries, 
	breaklines=true,
	tabsize=2,
	xleftmargin=\fboxsep,
	xrightmargin=-\fboxsep,
	numbers=left,
	numberstyle=\tiny\color{gray},
	stepnumber=1,
	numbersep=5pt,
	frame=bt,
	stringstyle=\color{darkred},
	showstringspaces=false,
	rulecolor= \color{gray},
	emph=[1]%
	{%  
	    then, not, for, return%
	},
	emphstyle=[1]{\color{darkblue}\bfseries},
	emph=[2]%
	{%  Datatypes
	    %
	},
	emphstyle=[2]{\color{darkblue}\bfseries},
	emph=[3]%
	{%  
	    %
	},
	emphstyle=[3]{\color{darkred}\bfseries},
	literate=%
	{Ö}{{\"O}}1
	{Ä}{{\"A}}1
	{Ü}{{\"U}}1
	{ß}{{\ss}}2
	{ü}{{\"u}}1
	{ä}{{\"a}}1
	{ö}{{\"o}}1
}
\providecommand{\tabularnewline}{\\}

\usepackage{fancyhdr}
\pagestyle{fancy}
\usepackage{lastpage}
\makeatletter

\lhead{\textbf{\@title} \\ \@author}
\chead{}
\rhead{\@date \\ \thepage \ von \pageref{LastPage} }
\cfoot{}

\renewcommand{\maketitle}{}
\newcommand{\utilde}[1]{\underaccent{\tilde}{#1}}
\renewcommand{\familydefault}{\sfdefault}
 
\author{Tobias Höppner}
\title{HA - Gummi-Übung 08.}
\date{Wintersemester 2014/2015}

\begin{document} 
\maketitle 

\subsubsection*{45. Wege mit Transportkapazität, 10 Punkte}
Lösung: Der Algorithmus orientiert sich an dem Prim Algorithmus zum erstellen eines minimalen Spannbaums.
Jedoch brechen wir früher ab, wenn wir am Knoten $B$ angekommen sind. Da wir dann den optimalen Weg von $A$ nach $B$ gefunden haben.\\
\begin{compactitem}
\item[-] Wähle Knoten $A$ als Graph $S$
\item[-] Solange $S$ noch nicht $B$ enthält:
\begin{compactitem}
\item[-] Wähle Kante $(x,y)$ mit dem minimalen Gewicht $w_{xy}$ aus, die einen neuen Knoten $T$ mit $S$ verbindet
\item[-] Füge $T$ und Kante $(x,y)$ zu $S$ hinzu\\
\end{compactitem}
\end{compactitem}
Die Kanten werden bei Entdeckung eines neuen Knotens in eine Prioritätswarteschlange einsortiert. Die Prioritätswarteschlange liefert immer die Kante mit optimalen Gewicht.\\
Der Algorithmus bricht ab, wenn der Knoten $B$ im Graphen $S$ enthalten ist. Anschließend muss noch ein Weg von $A$ nach $B$ in $S$ gefunden werden.\\

\subsubsection{Korrektheit}
Der Algorithmus nimmt immer nur die kleinste Kante, um von den bereits entdeckten Knoten zu einem neuen Knoten zu kommen. Der entstehende Graph ist mindestens ein Teilgraph des minimalen Spannbaums, der gegebenen Knotenmenge. Im schlimmsten Fall entsteht ein minimaler Spannbaum, dessen Korrektheit bereits bekannt ist.

\subsubsection*{Laufzeit}
Die Laufzeit ist stark von der verwendeten Prioritätswarteschlange abhängig. Die optimale Struktur für die Warteschlange bietet der Fibonacci Heap, mit der die Laufzeit optimal wird. Die Gesamtlaufzeit des Algorithmus beträgt $\mathcal{O}(|E| + |V| \log |V| + |S|)$ für das finden eines Weges von $A$ nach $B$.

\subsubsection*{48. Wechselgeld, 10 Punkte}
\begin{compactitem}
\item[b)] Beweisidee: Die Münzwerte sind so konstruiert, dass bestimme Paare vielfache voneinander sind. Der Greedy-Algorithmus teilt immer den Betrag durch den gegeben Geldwert und wiederholt diesen Schritt mit dem Restbetrag und den verbleibenden Geldwerten, bis der Betrag vollständig abgebildet ist.\\ Der Greedyalgorithmus wird immer die optimale Lösung finden, solange die gegebenen Geldwerte Teiler untereinander sind.\\

Musterlösung im Tutorium.\\
Grundidee hier ist es die Fälle zu zeigen, bei der die Münzen nicht optimal zusammen gefügt sind:\\
1,5,10,50,100 - doppelt\\
(1,2,2), (2,2,2), (10,20,20), (20,20,20)\\

Daraus ergibt sich folgendes Lemma:\\
In der Optimallösung sind folgende Eigenschaften gegeben:
$\sum 1,2 \leq 4$\\
$\sum 1,2,5 \leq 9$\\
$\sum 1,2,5,10 \leq 19$\\
$\sum 1,2,5,10,20 \leq 49$\\
\dots $\leq 99$\\
\dots $\leq 199$\\

Behauptung: greedy = optimale Lösung.\\
Beweis per Widerspruch:\\
I.A.: $G=1,...,9$ per Hand ausrechnen\\
I.S.: $G \in \{ 10,...,19 \}$\\
Fall 1: OPT verwendet 10 $\Rightarrow$ I.V: auf $(G-10)$\\
Fall 2: OPT verwendet 10 nicht $\Rightarrow$ Widerspruch laut Lemma!\\
Das Schema nun für alle Teilmengen durchführen und damit stellt sich immer ein Widerspruch ein.\\

\item[c)] siehe B. 
\end{compactitem}
\end{document}
