\documentclass[ngerman,a4paper]{report}
\usepackage[ngerman]{babel}
\usepackage[T1]{fontenc}
\usepackage[utf8]{inputenc}
%\usepackage{MyriadPro}
\usepackage[scaled]{beramono}
\newcommand\Small{\fontsize{10.5}{10.5}\selectfont}
\newcommand*\LSTfont{\Small\ttfamily\SetTracking{encoding=*}{-20}\lsstyle}
\usepackage{microtype}
\usepackage{geometry}
\geometry{verbose,tmargin=3cm,bmargin=3cm,lmargin=3cm,rmargin=3cm}
\usepackage{listings}
\usepackage{stmaryrd}
\usepackage{paralist}
\usepackage{array}
\usepackage{color}
\usepackage{graphicx}
\usepackage{caption}
\usepackage{url}
\usepackage{amsmath}
\usepackage{amssymb}
\usepackage{accents}
\usepackage{tikz}
\usetikzlibrary{arrows}

%tikz
\tikzset{
  treenode/.style = {align=center, inner sep=0pt, text centered,
    font=\sffamily},
  arn_n/.style = {treenode, circle, white, font=\sffamily\bfseries, draw=black,
    fill=black, text width=3.5em},% arbre rouge noir, noeud noir
  arn_r/.style = {treenode, red, 
    text width=1.5em, very thick},% arbre rouge noir, noeud rouge
  arn_x/.style = {treenode, rectangle, draw=black,
    minimum width=0.5em, minimum height=0.5em}% arbre rouge noir, nil
}

% Code Listing Style
\definecolor{darkblue}{rgb}{0,0,.6}
\definecolor{darkgreen}{rgb}{0,0.5,0}
\definecolor{darkred}{rgb}{0.5,0,0}
\lstset{%
	language=Python, 
	basicstyle=\LSTfont,
	commentstyle=\color{darkgreen}, 
	keywordstyle=\color{darkblue}\bfseries, 
	breaklines=true,
	tabsize=2,
	xleftmargin=\fboxsep,
	xrightmargin=-\fboxsep,
	numbers=left,
	numberstyle=\tiny\color{gray},
	stepnumber=1,
	numbersep=5pt,
	frame=bt,
	stringstyle=\color{darkred},
	showstringspaces=false,
	rulecolor= \color{gray},
	emph=[1]%
	{%  
	    then, not, for, return%
	},
	emphstyle=[1]{\color{darkblue}\bfseries},
	emph=[2]%
	{%  Datatypes
	    %
	},
	emphstyle=[2]{\color{darkblue}\bfseries},
	emph=[3]%
	{%  
	    %
	},
	emphstyle=[3]{\color{darkred}\bfseries},
	literate=%
	{Ö}{{\"O}}1
	{Ä}{{\"A}}1
	{Ü}{{\"U}}1
	{ß}{{\ss}}2
	{ü}{{\"u}}1
	{ä}{{\"a}}1
	{ö}{{\"o}}1
}
\providecommand{\tabularnewline}{\\}

\usepackage{fancyhdr}
\pagestyle{fancy}
\usepackage{lastpage}
\makeatletter

\lhead{\textbf{\@title} \\ \@author}
\chead{}
\rhead{\@date \\ \thepage \ von \pageref{LastPage} }
\cfoot{}

\renewcommand{\maketitle}{}
\newcommand{\utilde}[1]{\underaccent{\tilde}{#1}}
\renewcommand{\familydefault}{\sfdefault}
 
\author{Tobias Höppner}
\title{HA - Beton-Übung 13.}
\date{Wintersemester 2014/2015}

\begin{document} 
\maketitle 

\subsubsection*{NP-Übung}
Teilgraphisomorphie $\in $ NP?

\subsubsection*{Partition}

Eingabe: $a_1,...,a_n \in \mathbb{N}$\\
Frage: $\exists I_i J $ Partition von $\{1,...,n\} \sum_{i \in I} a_i = \sum_{i \in J} a_i$ ?\\

\subsubsection*{79. Zollstock}
Teilmengensumme $\leq_p$ Partition\\
$f: \{a_1,...,a_n, b\} \rightarrow \{a_1,...,a_n, (b+1),(S-b+1)\}$, wobei $S := \sum_{i=1}^n a_i$\\
$f$ ist in polynomialzeit berechenbar, weil nur $S$ berechnet werden muss.\\
Beweis:\\
x-Ja Instanz $\Leftrightarrow$ f(x) Ja-Instanz\\
$\Rightarrow \exists I \subseteq \{1,...,n\}$:\\
$\sum_{i \in I} a_i = b$\\
$I' := I \cup \{n+2\}$\\
$I' := \{1,...,n+2\} \setminus I'$\\
$\sum_{i \in I'} a_i = b + S - b +1 = s+1$\\
$\sum_{i \in J'} a_i = \sum_{i=1}^{n+2} a_i - \sum_{i \in I'} a_i = s + (b+1)+(s-b+1)-(s+1) = s+1$\\
$\leftarrow f(x)$-Ja-Instanz:
$\exists I, J $ Partition von $\{1,...,n+2\}$\\
$\sum_{i \in I} a_i = \sum_{i \in J} a_i = S+1$\\
$a_{n+1} + a_{n+2} = s+2$\\
d.h. $(n+1)$ und $(n+2)$ in unterschiedlichen $I,J$\\
o.B.d.A: $(n+2) \in I$\\
$\sum_{i \in I} a_i - a_{n+2} = (s+1)-(s-b+1) = b$\\
$I' := I \setminus \{n+2\}$\\

Partition $\leq_p$ Zollstock\\
$f_ \{a_1,...,a_n\} \rightarrow $  Zollstockelemente in $T$ aufteilen\\
$T:=2\sum_{i=1}^n a_i$\\
Futterallänge $=T$
$x \in $ Partition $\iff f(x) \in$ Zollstock\\
$\Rightarrow x \in $ Partition\\
$\Rightarrow \exists I, J$ Partition von $\{1,...,n\}$\\
$\sum_{i \in I} a_i = \sum_{i \in J} a_i$\\
Falte Glieder in $I$ nach rechts.\\   
Falte Glieder in $J$ nach links.\\
$\Rightarrow$ Faltung passt in das Futteral. \\
$\Leftarrow$ $f(x) \in $ Zollstock\\
Sei eine Faltung gegeben, die in das Futteral passt $\Rightarrow$ erstes und letztes Glied passen übereinander. $\Rightarrow$ Mittelteil Anfang und Ende sind an der selben Stelle.\\
$I := \{i | a_i $ nach rechts gefaltet.$\}$\\
$J := \{i | a_i $ nach links gefaltet.$\}$\\
Zollstock $\in$ NP\\
Zeuge: Faltung\\
Algorithmus: Überprüfe die Faltung, merke linkesten und rechtesten Punkt. Abstand $\leq $ Futterallänge.\\

\subsubsection*{80.}
\begin{enumerate}
\item[a.)] $n-1=00000...1011$\\
$n=0000...1100$\\
Worstcase, wenn $n$ $k$-Bits hat.\\
$01111...1 = m-1$\\
$\underbrace{10.....0}_{k} = m$\\
$00000$\\
$00001 \leftarrow$ nach dem ersten Aufruf\\
$(b_{k-1},b_{k-2},b_{0}) = n$\\
$2^{k-1} \leq n < 2^k$\\
$\Rightarrow k = \lfloor \log_n \rfloor +1$\\
$T_m = $\# Bitoperationen im i-ten Schritt\\
$f(n) := \max_{1 \leq m \leq n} T_m $\\
$2 (\lfloor \log_n \rfloor +1 )$
\item[b.)] $0 \rightarrow 1$ genau einmal pro Aufruf.\\
$1 \rightarrow 0$ höchstens $n$-mal bei dem ersten $n$-Aufrufen.\\
$g(n) = \sum_{k=1}^n \frac{T_k}{n} \leq \frac{4n}{n} = 4$\\
\end{enumerate}
\end{document}
