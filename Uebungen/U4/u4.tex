\documentclass[ngerman,a4paper]{report}
\usepackage[ngerman]{babel}
\usepackage[T1]{fontenc}
\usepackage[utf8]{inputenc}
%\usepackage{MyriadPro}
\usepackage[scaled]{beramono}
\newcommand\Small{\fontsize{10.5}{10.5}\selectfont}
\newcommand*\LSTfont{\Small\ttfamily\SetTracking{encoding=*}{-20}\lsstyle}
\usepackage{microtype}
\usepackage{geometry}
\geometry{verbose,tmargin=3cm,bmargin=3cm,lmargin=3cm,rmargin=3cm}
\usepackage{listings}
\usepackage{stmaryrd}
\usepackage{paralist}
\usepackage{array}
\usepackage{color}
\usepackage{graphicx}
\usepackage{caption}
\usepackage{url}
\usepackage{amsmath}
\usepackage{accents}
\usepackage{tikz}
\usetikzlibrary{plotmarks}

% Code Listing Style
\definecolor{darkblue}{rgb}{0,0,.6}
\definecolor{darkgreen}{rgb}{0,0.5,0}
\definecolor{darkred}{rgb}{0.5,0,0}
\lstset{%
	language=[x86masm]Assembler, 
	basicstyle=\LSTfont,
	commentstyle=\color{darkgreen}, 
	keywordstyle=\color{darkblue}\bfseries, 
	breaklines=true,
	tabsize=2,
	xleftmargin=\fboxsep,
	xrightmargin=-\fboxsep,
	numbers=left,
	numberstyle=\tiny\color{gray},
	stepnumber=1,
	numbersep=5pt,
	frame=bt,
	stringstyle=\color{darkred},
	showstringspaces=false,
	rulecolor= \color{gray},
	emph=[1]%
	{%  
	    then, not, for, return%
	},
	emphstyle=[1]{\color{darkblue}\bfseries},
	emph=[2]%
	{%  Datatypes
	    %
	},
	emphstyle=[2]{\color{darkblue}\bfseries},
	emph=[3]%
	{%  
	    %
	},
	emphstyle=[3]{\color{darkred}\bfseries},
	literate=%
	{Ö}{{\"O}}1
	{Ä}{{\"A}}1
	{Ü}{{\"U}}1
	{ß}{{\ss}}2
	{ü}{{\"u}}1
	{ä}{{\"a}}1
	{ö}{{\"o}}1
}
\providecommand{\tabularnewline}{\\}

\usepackage{fancyhdr}
\pagestyle{fancy}
\usepackage{lastpage}
\makeatletter

\lhead{\textbf{\@title} \\ \@author}
\chead{}
\rhead{\@date \\ \thepage \ von \pageref{LastPage} }
\cfoot{}

\renewcommand{\maketitle}{}
\newcommand{\utilde}[1]{\underaccent{\tilde}{#1}}
\renewcommand{\familydefault}{\sfdefault}
 
\author{Tobias Höppner}
\title{HA - Gummi-Übung 04. }
\date{Wintersemester 2014/2015}

\begin{document} 
\maketitle 

\subsection*{23. Bestimmen der Mehrheit, 10 Punkte}

\begin{enumerate}
\item[\textbf{a)}] nix, weil 0 Punkte. :)
\item[\textbf{b)}] \textbf{Idee:}\\
Liste durchlaufen, Histogramm(Wert, Anzahl) erstellen, Histogramm durchlaufen, wenn Wert Bedingung \lstinline!B! erfüllt ausgeben
\begin{lstlisting}
in: e[0,..,i]					; Werte
var: i = 0						; Laufvariable
array: hist = []			; Leeres Array, mit 0 iniziert
var: n = len(a) 			; laenge von Eingabe
var: B = (n+1)/2 +1		; Abschlussbedingung
; Histogramm befüllen
for i=0, i<n, i++:
	;eingelesener Wert entspricht der Position im Histogramm
	hist[e[i]]++
; Histogramm durchlaufen
for val in hist:
	; Wert erfüllt Bedingung
	if val >= B:
		; Wert zur Ausgabe hinzufügen
		out.append(val)
return out
\end{lstlisting}
\subsubsection*{Laufzeit}
Argumentativ: $\mathcal{O}(c*n)$, mit $n=$ Länge der Eingabe.\\
Zeilen 1-5: konstante Laufzeit\\
Zeile 7-9: $\mathcal{O}(n)$\\
Zeile 11-15: $\mathcal{O}(l(hist))$\\

Obere und untere Schranke unterscheiden sich nicht wesentlich und sind abhängig in der Größe vom Histogramm. Allgemein reicht es allerdings aus sich die Eingabe nur einmal $\Theta(n)$ anzusehen. Im schlimmsten Fall (alles verschiedene Werte) beträgt die Laufzeit $\mathcal{O}(2n)$.\\
Im schlimmsten Fall könnte man die Laufzeit optimieren in dem die Größe (Anzahl der Einträge) des Histogramms im Bezug zur Bedingung betrachtet.\\ 
\item[\textbf{c)}] Nehme Algorithmus aus \textbf{b)} und wähle \lstinline!B = n*0.33!.
\end{enumerate}




\subsection*{25. Lokales Maximum in Bäumen, 10 Punkte}

\end{document}
