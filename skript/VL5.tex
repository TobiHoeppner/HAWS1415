\section{Rekursion (Fortsetzung) \tiny (Vorlesung 5 am 31.10.)}
\subsection{ Motivation Master-Theorem}
\begin{align*}
T(n) &= \underbrace{ T(\frac{n}{b}) }_{T( \lfloor \frac{n}{b} \rfloor) + \dots + T(\lceil \frac{n}{b}\rceil)}* a + f(n)
\end{align*}
Für Probleme $\leq n_0$ wird das Problem irgendwie direkt gelöst.\\
Startbedingung: $1 \leq T(n) \leq M$ für $n\leq n_0$\\

In der Praxis muss man natürlich irgendwann das $n_0$ ausrechnen und kann nicht beliebig lange aufteilen.\\
Die Konstanten $a \geq 1$ und $b > 1$ müssen erfüllt sein und außerdem müssen wir fordern:
\begin{align*}
\lceil \frac{n}{b} \rceil &\leq n-1 \text{ für } n > n_0\\
\Leftrightarrow \frac{n}{b} &\leq n-1\\
n(1-\frac{1}{b}) &\geq 1\\
n \geq \frac{b}{b-1}\\
\Rightarrow n_0 &\geq \frac{b}{b-1}
\end{align*}
sonst werden die Probleme nicht kleiner und die Rekursion kann nicht gelöst werden.\\
$n \log_b n$ Elemente
$\begin{cases}
1 \text{ Problem der Größe } n & \text{Aufwand }  1f(n)  n^k \text{ Annahme }f(n)=n^k\\
2 \text{ Probleme der Größe } \frac{n}{b} &\text{Aufwand }  a*f(\frac{n}{b})  a(\frac{n}{b})^k\\
3 \text{ Probleme der Größe } \frac{n}{b^2} &\text{Aufwand }  a^2*f(\frac{n}{b^2})  a^2(\frac{n}{b})^k\\
\vdots & \vdots \\

\end{cases}$
\subsection*{Beispiel: Mergesort}
\begin{align*}
a = b = 2\\
\gamma = \log_2 2  =1\\
\end{align*}

\subsection{Master-Theorem für divide and conquer-Rekursion}
\begin{align*}
a \geq 1, b > 1, M, n_0 \geq 1 (\frac{n_0}{b} \leq n_0 -1)\\
f(n),T(n) \text{Funktionen auf den natürlichen Zahlen}\\
f(n) \geq 0\\
\end{align*}
Es gelten die Rekursionsbedingungen 
\begin{align*}
T(n) & \leq a T(\lceil \frac{n}{b} \rceil) + f(n) & (n > n_0)\\
T(n) & \geq a T(\lfloor \frac{n}{b} \rfloor) + f(n) & (n > n_0)\\
1 & \leq T(n) \leq M\\
\end{align*}
Dann definieren wir den kritischen Exponenten
\begin{align*}
n = \log a > 0
\end{align*}
\begin{enumerate}
\item[(-)] Wenn $f(n) = \mathcal{O}(n^{\gamma - \epsilon})$ für ein $\epsilon > 0$, dann\\
% im Kasten, mittig!
$T(n) = \Theta (n^\gamma)$
\item[(=)] Wenn $f(n) = \Theta(n^\gamma)$ ist, dann \\
% kasten, mittig
$T(n) = \Theta (n^\gamma \log n)$
\item[(+)] Wenn $f(n) = \Theta (n^{\gamma + \epsilon})$ für ein $\epsilon > 0$ ist oder wenn die Reularitätsbedingung erfüllt ist $\exists c < ^1$:\\
(*) $a.f(\rceil \frac{n}{b}) \lceil < c.f(n)$ für alle $n>n_0$\\
dann gilt:
%kasten, mittig
$T(n) = \Theta(f(n))$
\end{enumerate}
\subsubsection{Bemerkungen}
\begin{enumerate}
\item Wenn $f$ monoton ist, dann gelten die Schlussfolgerungen auch für beliebig gemischtes Auf- und Abrunden.
\item Mit (*) kann man auch Funktionen wie $f(n) = 2^n$ oder $f(n) = 2^{\sqrt{n}}$ erfassen.
\item $\Omega (n^{\gamma + \epsilon})$ im Fall (+) reicht leider nicht.
\item $f(n) = n\log n, \gamma = 1$ wird nicht erfasst.
\end{enumerate}
\subsection{Beweis: Master-Theorem}
\begin{enumerate}
\item[a.)] Wir betrachten die oberen Schranken für die Fälle (-) und (=)
\begin{enumerate}
\item Ersetze $f(n)$ durch die oberen Schranke $\underline{u}.n^k$\\
$f(n) \leq u.n^k$
Finde eine Funktion $P(n)$ mit %P = T mit Dach\\
$(***) P(n) \geq  a P(\lceil \frac{n}{b} \lceil) + u n^k$ für $n \geq n_0$\\
und $P(n) \geq M$ für $n \geq n_0$\\
Dann ergibt sich durch vollständige Induktion: $T(n) \leq P(n)$\\
Basis: $(n \leq n_0) $\\%checkmark
\begin{align*}
T(n) &\leq a T(\lceil \frac{n}{b} \rceil) + f(n) \leq \text{(I.V.)}\\
& \leq a P(\lceil \frac{n}{b} \rceil) + f(n)\\
& \leq a P(\lceil \frac{n}{b} \rceil) + u n^k \leq P(n)\\
\end{align*}
\item Verschiebung des Definitionsbereiches, um $\lceil \rceil$ los zu werden.\\
$v = \frac{b}{b-1} \Rightarrow -\frac{v}{b} = 1-v$\\
$P(n) = T'(n-v)$ bzw. $T'(n) = P(n+v)$\\
$T' ist jetzt auf \mathbb{R}_{>0}$ definiert.\\
Wir bestimmten dann $T'$ so, dass \\
$(**) T'(n) = a T'(\frac{n}{b}) + u' n^k$ ($u$ ist eine Konstante)\\
\textbf{Behauptung:} aus $(**)$ folgt $(***)$, falls $T'$ monoton wächst
\begin{align*}
\underbrace{P(n)} &\geq a P(\lceil \frac{n}{b} \rceil) + un^k\\
\text{L.S.}&=P(n) = T'(n-v) = a T'(\frac{n}{b} - \frac{v}{b})+u'(n-v)^k\\
\text{R.S.}&= a P(\lceil \frac{n}{b} \rceil) + un^k\\
&= a T'(\lceil \frac{n}{b} \rceil - v) + un^k\\
&< a T'(\frac{n}{b} +1 - v) + un^k\\
&= a T'(\frac{n-v}{b}) + un^k
\end{align*} 
Jetzt müssen wir nur noch $u'$ so wählen, dass $u'(n-v)^k \geq u n^k$ für $n \geq n_0 u'\geq u \frac{n_{0}^k}{(n_0-v)^k}$\\
Lösen von $(**)$ durch Ansatz:
\begin{enumerate}
\item[Fall (-)] $k = \gamma - \epsilon: T'(n) = D n^\gamma + En^k$
Einsetzen in $(**)$
\begin{align*}
Dn^\gamma + En^k &= aD(\frac{n}{b})^\gamma + aE(\frac{n}{b})^k + u'n^k\\
&= Dn^\gamma \underbrace{\frac{a}{b^\gamma}}_{1} + n^k(aE \frac{1}{b^k} + u')\\
\\
E(1-\frac{a}{b^k} = u', E = \frac{u'}{1-\frac{a}{b^2}})&\\
E(1-\frac{b^\gamma}{b^{\gamma - \epsilon}}) &= u'\\
E (1-b^\epsilon) = u'\\
\underline{E} = \frac{-u'}{b^\epsilon -1} < 0\\
\end{align*}
$D$ ist noch frei: Wähle $D$ groß genug, dass $P(n) = T'(n-v) = D(n-v)^\gamma + E(n-v)^k \geq M$ für $n \leq n_0$ ist.
\item[Fall (=)]
\begin{align*}
T'(n) &= D n^\gamma + En^\gamma \log_b n\\
\dots \Rightarrow E &= u'\text{, D bleibt frei. - D groß genung.}
\end{align*}
\item[Ergebnis im Fall (-)]$T(n) \leq D(n-v)^\gamma + E(n-v)^{\gamma-k} = \mathcal{O(n^\gamma)}$
\item[Ergebnis im Fall (=)] $= \mathcal{O(n^\gamma \log \frac{n}{\gamma})}$
\end{enumerate}
\end{enumerate}
\end{enumerate}

